% interactnlmsample.tex
% v1.05 - August 2017

\documentclass[]{interact}

\usepackage{epstopdf}% To incorporate .eps illustrations using PDFLaTeX, etc.
\usepackage[caption=false]{subfig}% Support for small, `sub' figures and tables
%\usepackage[nolists,tablesfirst]{endfloat}% To `separate' figures and tables from text if required
%\usepackage[doublespacing]{setspace}% To produce a `double spaced' document if required
%\setlength\parindent{24pt}% To increase paragraph indentation when line spacing is doubled

\usepackage[numbers,sort&compress]{natbib}% Citation support using natbib.sty
\bibpunct[, ]{[}{]}{,}{n}{,}{,}% Citation support using natbib.sty
\renewcommand\bibfont{\fontsize{10}{12}\selectfont}% Bibliography support using natbib.sty
\makeatletter% @ becomes a letter
\def\NAT@def@citea{\def\@citea{\NAT@separator}}% Suppress spaces between citations using natbib.sty
\makeatother% @ becomes a symbol again

\theoremstyle{plain}% Theorem-like structures provided by amsthm.sty
\newtheorem{theorem}{Theorem}[section]
\newtheorem{lemma}[theorem]{Lemma}
\newtheorem{corollary}[theorem]{Corollary}
\newtheorem{proposition}[theorem]{Proposition}

\theoremstyle{definition}
\newtheorem{definition}[theorem]{Definition}
\newtheorem{example}[theorem]{Example}

\theoremstyle{remark}
\newtheorem{remark}{Remark}
\newtheorem{notation}{Notation}

\begin{document}

\articletype{RESEARCH ARTICLE}% Specify the article type or omit as appropriate

\title{
	HTCondor Universes\\
	A Systematic Mapping Study
}


\author{
\name{Juan Esteban Parra Parra\textsuperscript{a}\thanks{CONTACT Juan Esteban Parra Parra. Email: [jep.parra.parra@uqvirtual.edu.co]}, Juan Esteban Castaño Osma\textsuperscript{a}, Luis Eduardo Sepúlveda Rodríguez\textsuperscript{a}, and Christian Andres Candela Uribe\textsuperscript{a}}\affil{\textsuperscript{a}Universidad del Quindío, Carrera 15 \#12N, Armenia, Quindío, Colombia}
}

\maketitle

\begin{abstract}
	Distributed computing systems such as HTCondor have gained acceptance in the scientific research space. HTCondor is a job scheduling distrubuted system with various execution contexts called universes. While widely used in the scientific community, few efforts have been made to classify its applications across IT domains and academic functions like teaching, research and industry collaboration. This work presents a systematic mapping study that seeks to classify existing HTCondor universe implementations, providing a taxonomy that offers a structured view of the current landscape and facilitates informed decision-making.
\end{abstract}

\begin{keywords}
	HTCondor; Distributed Computing; HTCondor Universes; Condor; Systematic Mapping Study
\end{keywords}


###############################################################################################

\section{Introducción}
\IEEEPARstart{E}{l} siguiente es el inicio de nuestra introducción
%lorem
Ipsum nostrud exercitation eiusmod dolor. Cillum adipisicing est cupidatat incididunt aliquip dolor culpa dolore id culpa incididunt elit. In est occaecat nulla culpa. Officia anim aute in irure proident fugiat aliqua esse ipsum aliquip minim.

Aliquip commodo nulla consectetur laborum culpa ipsum anim eu labore amet culpa veniam. Ad quis consequat et commodo eiusmod dolore ut ut enim duis laborum. Sint nulla eiusmod laboris ipsum sint cillum in labore irure ipsum. Do fugiat qui enim duis pariatur deserunt ex occaecat. Ipsum aute id Lorem eu esse anim aute culpa aliquip reprehenderit sit duis veniam anim. Aute deserunt aute id ea officia magna veniam aliqua ut occaecat ullamco proident.

Voluptate officia elit ullamco est pariatur. Lorem ad nulla pariatur dolore nulla. Ullamco pariatur adipisicing commodo aliquip.

Fugiat eu pariatur occaecat magna excepteur eu non exercitation minim anim voluptate amet. Veniam consectetur nostrud reprehenderit sit pariatur eu adipisicing est elit est. Elit aliquip labore ullamco qui cupidatat ut eu non nulla ipsum irure. Incididunt ut excepteur magna proident elit eu laboris nostrud. Cillum pariatur est magna quis fugiat dolor commodo non sint amet exercitation. Ex ea labore non officia.

Dolor cillum veniam reprehenderit id est consectetur reprehenderit voluptate ea in aliquip mollit dolore amet. Sunt mollit eiusmod est cupidatat proident quis velit eu veniam eiusmod et adipisicing dolor. Officia commodo et commodo sint aliquip quis nostrud minim laborum culpa ut. Quis adipisicing labore do esse id minim adipisicing mollit aliqua tempor. Ea ipsum nostrud fugiat sunt Lorem cupidatat do ipsum sint minim consequat laboris pariatur.

Do sit officia qui pariatur reprehenderit consectetur magna proident ad culpa esse amet cillum. Laborum ullamco enim aliqua enim fugiat laboris sint cillum. Enim id velit aliquip laboris veniam nisi excepteur fugiat consequat ea excepteur duis Lorem velit. Anim adipisicing est laborum aliquip ut minim aliqua minim minim pariatur nisi ipsum tempor. Sit qui labore enim voluptate velit deserunt. Non officia incididunt amet qui ipsum.








\end{document}
