\documentclass[journal]{IEEEtran}

% Paquete para soporte del idioma español
% Importante para que la compilación de términos generados automaticamente se haga en español.
\usepackage[spanish]{babel}
% Use natbib for compressed citations
\usepackage[numbers,compress]{natbib}

\usepackage[utf8]{inputenc}
\usepackage{array}
\usepackage{booktabs}


\usepackage{multirow, array} % para las tablas
\usepackage{graphicx}

\usepackage{tabularx}

\usepackage{titlesec}
% Force line breaks after subsubsections
\titleformat{\subsubsection}
  {\normalfont\normalsize\itshape}{\thesubsubsection}{1em}{}[\vspace{1ex}]
  \titlespacing{\subsubsection}
  {0pt} % left indentation
  {3ex plus 1ex minus 0.5ex} % space before de subsubsection
  {1ex plus 0.5ex minus 0.2ex} % space after de subsubsection
  % ------ explaining the latter part: -----------
  % 2ex = Base spacing (2 times the height of letter "x" in current font)
  % plus 0.5ex = Maximum additional space LaTeX can add if needed
  % minus 0.2ex = Maximum space LaTeX can remove if needed


% --- PAQUETE PARA HIPERVÍNCULOS ---
% La finalidad de su uso en este proyecto es otorgar enlaces a las citas hechas a traves del documento.
\usepackage{hyperref} % <---- Load this package LAST!
\hypersetup{
    colorlinks=true,
    linkcolor=blue,    % Color for internal links (e.g., sections)
    citecolor=blue,    % Color for citations
    filecolor=blue,    % Color for file links
    urlcolor=blue,     % Color for external URLs
}
% ---------------------------------

% Redefinir el nombre de las palabras clave a español
\renewcommand{\IEEEkeywordsname}{Palabras Clave}

% Redefinir el nombre de las tablas al español - CORRECT WAY with babel
\addto\captionsspanish{\renewcommand{\tablename}{Tabla}}

% Comando artesanal para escribir texto con dos propiedades 
% italica y negrilla. 
% uso -> \bolditalic{su texto aquí!!!!!!!}
\newcommand{\bolditalic}[1]{\textbf{\textit{#1}}}



% Estructura del documento como referencia para el futuro
%\part           Optional - for very large documents
%  \chapter      Top-level division
%    \section
%      \subsection
%        \subsubsection
%          \paragraph
%            \subparagraph


% !!! Importante 
% Para la nomemclatura de archivos
% 1) Aquellos archivos (no incluye directorios) que empiezan con "00" deben ser considerados como 
%    el archivo "main" de esa sección

\begin{document}

\title{
	Universos HTCondor en la infraestrutura del Grupo GRID de la
	Universidad del Quindío \\
	Estudio de mapeo sistematico
}

\author{
	Juan Esteban Parra Parra,~\IEEEmembership{Estudiante de pregrado, Universidad del Quindío}, \\
	Juan Esteban Castaño Osma,~\IEEEmembership{Estudiante de pregrado, Universidad del Quindío} y \\
	Luis Eduardo Sepúlveda Rodríguez,~\IEEEmembership{PhD, Universidad del Quindío}
}

% The paper headers
%!TODO Colocar el nombre de la revista una vez lo sepamos.
\markboth{Revista LA REVISTA,~Vol.~XX, No.~YY, Agosto~2025}{Header de la derecha}


\maketitle

%!TODO Abstract.
\begin{abstract}
	Commodo enim magna ea nulla commodo cillum quis sit nisi et. Minim pariatur est quis nisi. Incididunt nisi duis nisi laborum est mollit fugiat. Aliqua mollit do ut est do minim non ullamco laboris occaecat. Deserunt est enim exercitation aliqua voluptate est sunt magna amet ex exercitation.
\end{abstract}


%!TODO Keywords
\begin{IEEEkeywords}
	HTCondor, Computación distribuída, Universos HTCondor, Condor
\end{IEEEkeywords}


% ########################### SECCIONES ###########################

%Seccion introducción
%sección I
\section{Introduction}
\label{sec:introduccion}
Scientific computing addresses complex problems that exceed the scope of manual analytical approaches~\cite{landau01}. Computers are indispensable in this context, as many real-world problems involve levels of difficulty or data scale that make purely analytical approaches infeasible; nevertheless, making use of computational resources allows these problems to be effectively addressed~\cite{landau01}.

However, many scientific problems cannot be effectively addressed by a single computer due to their computational complexity. Certain problems are computationally infeasible to solve on a single machine due to factors such as their inherent complexity or the size of their input datasets. For this reason, researchers employ distributed computing tools to produce results in a reasonable time. Some of these tools belong to the realm of High Throughput Computing (HTC), a paradigm whose goal is to maximize the volume of results produced over extended periods of time~\cite{juve-01} and which has garnered interest in educational contexts~\cite{Senol-01}.

Within this domain, HTCondor, developed at the University of Wisconsin–Madison, has emerged as a workload management system designed specifically for compute-intensive applications~\cite{chang-01, htcondor-description}. HTCondor allows users to submit tasks to a pool, where the system autonomously manages resource allocation, scheduling, and distribution across nodes. Its scheduling mechanism follows a bidirectional policy: both resource owners and job submitters may define conditions and preferences that influence how and where tasks are executed~\cite{htcondor-description}.

HTCondor organizes computational jobs into distinct execution environments, known as universes. At the time of writing, the available universes include: vanilla, grid, java, scheduler, local, parallel, vm, container, and docker. This diversity reflects the system’s adaptability, ranging from traditional computing tasks to virtualized and containerized environments.

Despite this versatility, the literature lacks a systematic classification that clarifies the contexts in which HTCondor universes are applied and their broader impact. To address this gap, this paper presents a systematic mapping study (SMS) with two objectives: first, to classify works across technological domains such as distributed and parallel computing, software development, virtualization, containerization, and networking, among others; and second, to identify and categorize works that leverage HTCondor universes as tools supporting academic functions (AF) such as research, teaching, and industry collaboration.

The remainder of this document is structured as follows:
Section~\ref{sec:motivacion} outlines the motivation for this work.
Section~\ref{sec:trabajos-relacionados} reviews related studies.
Section~\ref{sec:metodo-revision} describes the method employed to conduct the SMS.
Section~\ref{sec:analisis-discusion} presents the analysis and discussion of the study.
Section~\ref{sec:amenazas-validez} discusses threats to validity, and finally,
Section~\ref{sec:conclusiones} provides the conclusions.

%Sección motivación
\section{Motivation} \label{sec:motivacion}
Despite the adoption of HTCondor across various scientific and technological domains, the literature reveals a notable scarcity of studies regarding the specific use of its execution universes.
The absence of a systematic classification of works related to HTCondor universes hinders informed decision-making for their implementation and limits the effective transfer of knowledge among different user communities. Furthermore, existing literature does not provide a consolidated perspective on how these universes contribute to strengthening research, teaching, and industry collaboration within an academic context.
This lack of systematized knowledge motivated the present systematic mapping study, with the aim of offering a structured and comprehensive overview that supports informed decision-making and promotes the optimal use of HTCondor across diverse application scenarios.


%Sección trabajos relacionados
%sección III
\section{Trabajos Relacionados}\label{sec:trabajos-relacionados}

No se identificaron estudios previos que compartan los objetivos de
esta investigación. Adicionalmente, la literatura carece de trabajos
que examinen específicamente los universos de HTCondor, confirmando
la necesidad del presente mapeo sistemático.

%Sección método de revisión
%sección IV
\section{Método de revisión}
Pariatur incididunt commodo exercitation ad ex ex anim cillum nulla incididunt adipisicing irure et esse. Laborum enim amet eiusmod deserunt aute non minim duis aute est. Non magna elit cupidatat nulla minim duis est enim. Sint magna velit esse aliquip ea.

Esse duis quis velit non reprehenderit minim cupidatat exercitation fugiat labore magna enim. Ut tempor reprehenderit enim cillum duis commodo consectetur sint consectetur pariatur consequat ea enim. Sunt aliqua qui esse incididunt. Aliquip aute veniam dolor labore eu est ad aute consequat Lorem. Ut Lorem duis anim qui qui.


%subsección 4.1
\subsection{Planeación}
En esta etapa, se estableció el propósito general de la investigación y se definieron las metas, así como las preguntas de investigación, métricas, criterios de clasificación, criterios de inclusión/exclusión y criterios de calidad de los estudios. Ver Figura~\ref{fig:etapa1}. \\

\begin{figure*}[tbp]
    \centering
    \includegraphics[width=0.8\textwidth]{resources/images/planeacion/etapa1.png}
    \caption{Composición de la etapa de planeación}\label{fig:etapa1}
\end{figure*}

\subsubsection{Objetivos}
\mbox{}\\
Teniendo en cuenta los aspectos descritos en la sección de motivación, se definieron 2 metas generales para la revisión sistemática de la literatura que se presentan en el cuadro~\ref{tab:metas}.

\begin{table}[htbp]
    \centering
    \begin{tabular}{>{\centering\arraybackslash}m{1cm} >{\arraybackslash}m{7cm}}
        \hline
        \textbf{Goal} & \textbf{Description} \\
        \hline
        M1 & Identificar trabajos relacionados con VBC en proyectos de docencia, investigación y extensión. \\
        \\
        M2 & Clasificar trabajos relacionados con VBC en los dominios de desarrollo de software, pensamiento computacional, computación paralela, análisis de datos, inteligencia artificial, redes computacionales, infraestructura de TI, HPC, entre otros. \\
        
        \hline
    \end{tabular}
    \caption{Metas del estudio}\label{tab:metas}
\end{table}





\subsubsection{Pregunta de investigación}
\mbox{}

\begin{table}[tbp]
    \scriptsize % reduce tamaño del texto
    \centering
    \renewcommand{\arraystretch}{1.3}
    \begin{tabularx}{\columnwidth}{>{\centering\arraybackslash}m{0.18\columnwidth} >{\RaggedRight\arraybackslash}X}
        \hline
        \textbf{Aspecto} & \textbf{Descripción} \\
        \hline
        Población & Trabajos relacionados con la VBC aplicadas en diversos dominios de TI con un énfasis en la educación, investigación y extensión. \\
        Intervención & Identificación y clasificación de los trabajos en VBC en los dominios de TI establecidos. \\
        Comparación & 
        1. Se comparan los proyectos que han hecho uso de la VBC para determinar cuáles han tenido mayor tasa de éxito expresado por los autores en cada dominio de TI. \newline
        2. Se analiza el impacto de la VBC en proyectos de docencia, investigación y extensión en comparación con otras soluciones tecnológicas. \\
        Salida & Estructura de clasificación de los trabajos relacionados con las VBC en cada dominio de TI que han impactado en proyectos de docencia, investigación y extensión. \\
        Contexto & Docencia, investigación y extensión con apropiación de los dominios de TI en forma de VBC. \\
        \hline
    \end{tabularx}
    \caption{Aspectos del modelo PICOC}\label{tab:PICOC}
\end{table}

\begin{table*}[!t]
\centering

\renewcommand{\arraystretch}{1.4}
\begin{tabularx}{\textwidth}{>{\centering\arraybackslash}m{0.05\textwidth} >{\centering\arraybackslash}m{0.05\textwidth} >{\RaggedRight\arraybackslash}X >{\RaggedRight\arraybackslash}X}
\toprule
\textbf{Meta} & \textbf{Pregunta} & \textbf{Descripción} & \textbf{Motivación} \\
\midrule
G1 & Q1 & ¿Cuáles son los trabajos relacionados con tecnologías de virtualización basadas en contenedores (VBC) que podrían impactar positivamente proyectos de docencia, investigación y extensión? & La transversalidad que ofrece la VBC, gracias a su reproducibilidad de entornos, permite estimular diferentes aristas de la sociedad. Su naturaleza facilita el transporte de soluciones de TI entre diferentes entornos, generando que una innovación en cualquier dominio social impacte directamente en otro. \\
\midrule
G2 & Q2 & ¿Cuáles son los principales trabajos relacionados con las tecnologías de virtualización basadas en contenedores (VBC) que podrían contribuir en los diversos dominios de TI, entre los que pueden ser desarrollo de software, pensamiento computacional, computación paralela, análisis de datos, inteligencia artificial, redes computacionales, infraestructura de TI, HPC, entre otros? & Se busca proporcionar una base sólida para investigadores, docentes y profesionales interesados en comprender el estado del arte actual en relación con las VBC, además del alcance y las aplicaciones de estos trabajos sin necesidad de un análisis profundo. \\
\bottomrule
\end{tabularx}
\caption{Preguntas de investigación y su motivación}
\label{tab:preguntas}
\end{table*}

Este modelo permite establecer los aspectos de ``Población'', ``Intervención'', ``Comparación'', ``Salida'' y ``Contexto'' que sirven para situar el trabajo a realizar. Ver cuadro~\ref{tab:PICOC}. \\
\\
Teniendo en cuenta el modelo PICOC, se definieron las preguntas de investigación. Ver cuadro~\ref{tab:preguntas}.\\

\subsubsection{Métricas}
\mbox{}\\

\begin{table}[htbp]
\centering
\renewcommand{\arraystretch}{1.3}
\begin{tabularx}{\columnwidth}{>{\centering\arraybackslash}m{0.15\textwidth} >{\RaggedRight\arraybackslash}X}
\toprule
\textbf{Métrica} & \textbf{Descripción} \\
\midrule
M1 & Cantidad de trabajos identificados en cada dominio de TI. \\
M2 & Cantidad de trabajos que están incluidos en educación. \\
M3 & Cantidad de trabajos que están incluidos en investigación. \\
M4 & Cantidad de trabajos que están incluidos en extensión. \\
\bottomrule
\end{tabularx}
\caption{Métricas definidas para el análisis}
\label{tab:metricas}
\end{table}

Se definieron las métricas del estudio usando un enfoque cuantitativo de acuerdo con la estructura de clasificación. Los detalles de las métricas se presentan en el cuadro~\ref{tab:metricas}.
Los criterios determinados limitaron la validez de los documentos a tres años, buscando la actualidad en el estudio. Además, el tipo se limitó a estudios primarios, buscando un rigor mayor en la revisión por pares.\\

\subsubsection{Tópicos de investigación}
\mbox{}\\
Las preguntas de investigación y el modelo PICOC sirven como línea base para definir los tópicos de investigación que se consideran relevantes para el estudio. Estos tópicos son: \textit{Container-based virtualization}, \textit{Education}, \textit{Research}, \textit{Industry}. 
La definición de los tópicos de investigación se realizó teniendo en cuenta los dominios de TI que se consideraron relevantes para el estudio.\\

\subsubsection{Criterios de inclusión y exclusión}
\mbox{}\\
\begin{table*}[!t]
\centering
\renewcommand{\arraystretch}{1.4}
\begin{tabularx}{\textwidth}{>{\centering\arraybackslash}m{0.15\textwidth} >{\RaggedRight\arraybackslash}X >{\RaggedRight\arraybackslash}X}
\toprule
\textbf{Categoría} & \textbf{Inclusión} & \textbf{Exclusión} \\
\midrule
Campos & Abstract & -- \\
\midrule
Tipo de publicación & Journal articles and conference proceedings & Thesis and book chapters \\
\midrule
Área/Disciplina & Management, Computer Science, Information Technology and Management, Engineering & Areas not related to virtualization, Computer Science, and Information Technology and Management \\
\midrule
Período & Between 2022 to 2024 & Less than 2022 \\
\midrule
Idioma & English & -- \\
\bottomrule
\end{tabularx}
\caption{Criterios de inclusión y exclusión}\label{tab:criterios}
\end{table*}

Los criterios de inclusión y exclusión se definieron para garantizar que los estudios seleccionados sean relevantes para las preguntas de investigación y los objetivos del estudio. Los criterios se presentan en el cuadro~\ref{tab:criterios}.
Se definió un período de 3 años en busca de la actualidad de los estudios. Además, se limitó los estudios a artículos de revistas buscando un mayor rigor en la revisión por pares. Los estudios deben estar escritos en inglés y en las áreas de \textit{Computer Science} y \textit{Management}, \textit{Information Technology and Management}, \textit{Engineering} en busca de la calidad de los estudios. Finalmente, se excluyeron los estudios que no están relacionados con la VBC, que no son revisados por pares o que no están disponibles en línea.\\

\subsubsection{Criterios de calidad}
\mbox{}\\
Para finalizar la etapa de planeación, se definieron tres criterios de calidad. \\

El primer criterio de calidad es una adaptación del CVI (Content Value Index)~\cite{almanasreh2019evaluation} y~\cite{yaghmaei2003content}.
En este caso, los artículos se evaluaron para determinar si cumplen con los criterios de inclusión y exclusión definidos y si son relevantes para las preguntas de investigación. Se usó una escala cuantitativa de 0 a 5, donde 0 indica una baja relación con las metas del SMS y 5 indica una alta relación.
Ver formula~\ref{eq:cvi}. En esta formula, K es el número impar de evaluadores y f(n) es la frecuencia de respuestas para cada valor de la escala.\\

\begin{equation}
\label{eq:cvi}
CVI = \frac{\sum_{n=1}^{k} f(n)}{k}
\end{equation}

El segundo criterio de calidad es el número de citas de cada estudio de acuerdo con la fecha de publicación (A), el cual se denomina SCI (Scientific Citation Index). Ver formula~\ref{eq:sci}. En esta formula, C es el número de citas entre 2022 y 2024 y A es el tiempo de publicación del estudio. Así, un artículo publicado en 2024 con las misma cantidad de citas que un artículo publicado en 2022 tendrá un SCI más alto.\\

\begin{equation}
\label{eq:sci}
SCI = \frac{C}{A}
\end{equation}

El tercer criterio de calidad corresponde a la relación de los estudios con las preguntas de investigación. Este criterio se denomina IRRQ (Indice de relación con las preguntas de investigación). Ver formula~\ref{eq:irrq}. 

\begin{equation}
\label{eq:irrq}
IRRQ = \frac{N}{2}
\end{equation}

N corresponde al número de preguntas de investigación que el estudio responde. Este valor se divide en 2 porque es el número de preguntas de investigación definidas en la etapa de planeación. \\

%subsección 4.2
% subseccion 4.2
\subsection{Etapa 2: Búsqueda de Estudios}
Esta etapa presenta la estrategia de búsqueda usada en la revisión sistemática de la literatura. Esta estrategia se describe en detalle en las subsecciones~\ref{subsubsec:Definiendo la Estrategia de Busqueda} -- \ref{subsubsec:resultados-busqueda}. Ver figura~\ref{fig:etapa2}.

\begin{figure*}[tbp]
    \centering
    \includegraphics[width=0.8\textwidth]{resources/images/planeacion/estrategias-busqueda.png}
    \caption{Composición de la etapa de búsqueda de estudios}\label{fig:etapa2}
\end{figure*}
\mbox{}\\

%sub-subseccion 4.2.1 
\subsubsection{Definiendo la Estrategia de Búsqueda}\label{subsubsec:Definiendo la Estrategia de Busqueda}
\mbox{}\\
% Content for 4.2.1
Para la construcción de esta revisión de la literatura, se usó un enfoque híbrido. Con este enfoque se busca obtener mayor volumen de artículos indexados y con diferentes origenes. más allá de los proporcionados por las bases de datos.
En este sentido, se combinaron dos estrategias de búsqueda. La primer estrategia es la búsqueda en bases de datos y consiste en realizar una cadena de búsqueda automatizada en bases de datos académicas.~\cite{jalali2012systematic}.
La segunda estrategia es la denominada ``Bola de Nieve'' (Snowballing) y consiste en la búsqueda manual de artículos a partir de un conjunto base de artículos usando las referencias y las citas de los mismos. Esta estrategia se basa en la premisa de que los artículos relevantes citan otros artículos relevantes y, por lo tanto, permite encontrar artículos que no están indexados en las bases de datos académicas.~\cite{jalali2012systematic} y \cite{goodman1961snowball}.
\mbox{}\\
%sub-subseccion 4.2.2

\subsubsection{Estrategia de Búsqueda 1: Bases de Datos}
\mbox{}\\
Esta estrategia consta de 2 componentes. El primer componente es denominado ``Identificación de estudios''. Esto se enfoca en definir las palabras clave para construir las cadenas de búsqueda que conducen a completar las búsquedas en las bases de datos académicas.
El segundo componente es llamado ``Selección de estudios''. Se enfoca en aplicar varios criterios para refinar la búsqueda de resultados de estudios y así obtener el mayor valor del proceso de búsqueda. \\ \\

\begin{itemize}
    \item \textbf{Identificación de estudios:} En búsqueda de la viabilidad del estudio y por acuerdo de los autores, se limitó la búsqueda a cinco bases de datos académicas: \textit{ACM}, \textit{IEEE Xplore}, \textit{Springer}, \textit{Science Direct} y \textit{Taylor and Francis}. En esta parte del proceso es necesario establecer las palabras claves definidas antes y construir las cadenas de búsqueda específica para cada base de datos. Nuevamente se usó el modelo PICOC como guía metodológica para identificar términos claves o frases completas que se relacionan con las tecnologías de virtualización basadas en contenedores. En la construcción de estas cadenas de búsqueda se usaron sinónimos para ampliar el espectro de resultados. (Ver cuadro~\ref{tab:palabras-clave}).\\ 
    Las principales palabras clave identificadas fueron: \textit{Container-based virtualization}, \textit{Education}, \textit{Research}, \textit{Industry}. Para ampliar y refinar los resultados se usaron operadores booleanos como \textit{AND} y \textit{OR}. Además, se usaron comillas para buscar frases completas y paréntesis para agrupar términos relacionados. Finalmente, el conjunto de palabras clave seleccionadas para construir las cadenas de búsqueda se presenta en el cuadro~\ref{tab:keywords}.\\
    Para dirigir la búsqueda hacia la intercepción de los dominios de TI y la VBC se usó el operador booleano \textit{AND}. Una vez identificadas las palabras clave, se procedió con la construcción de las cadenas de búsqueda para cada base de datos, usando un proceso iterativo. El proceso de construcción de las cadenas de búsqueda consitió en realizar un proceso heúristico con las palabras clave, sinónimos y conceptos relacionados, haciendo uso de conjunciones y disyunciones conforme a las reglas de cada base de datos.\\ 
    Así, estas cadenas de búsqueda varían de acuerdo con las reglas de cada base de datos. Ver cuadro~\ref{tab:cadenas-busqueda}.\\

\end{itemize}

\begin{table}[tbp]
    \scriptsize % reduce tamaño del texto
    \centering
    \renewcommand{\arraystretch}{1.3}
    \begin{tabularx}{\columnwidth}{>{\centering\arraybackslash}m{0.18\columnwidth} >{\RaggedRight\arraybackslash}X}
        \hline
        \textbf{Aspecto} & \textbf{Descripción} \\
        \hline
        Población & VBC, Dominios de TI, Educación, Investigación, Extensión \\
        Intervención & Identificación, Clasificación \\
        Comparación & Tasa de éxito, Evidencia de uso \\
        Salida & Clasificación de trabajos relacionados con VBC en cada dominio de TI \\
        Contexto & Docencia, Investigación, Extensión \\
        \hline
    \end{tabularx}
    \caption{Palabras clave identificadas usando el modelo PICOC}\label{tab:palabras-clave}
\end{table}

\begin{table}[tbp]
    \scriptsize % reduce tamaño del texto
    \centering
    \renewcommand{\arraystretch}{1.3}
    \begin{tabularx}{\columnwidth}{>{\centering\arraybackslash}m{0.18\columnwidth} >{\RaggedRight\arraybackslash}X}
        \hline
        \textbf{Palabras clave} & \textbf{Sinónimos} \\
        \hline
        Container-based virtualization & Application virtualization, Docker, Lightweight Virtualization \\
        Education & Education System, Education Development, Higher Education \\
        Research & Research Group, Research Proposal \\
        Industry & IT Services, Technology Infrastructure, Cloud Computing \\
        \hline
    \end{tabularx}
    \caption{Palabras clave para la búsqueda en base de datos}\label{tab:keywords}
\end{table}




\newcolumntype{P}[1]{>{\raggedright\arraybackslash}p{#1}}

\begin{sidewaystable*}[htbp]
\centering
\scriptsize
\renewcommand{\arraystretch}{1.5}
\begin{adjustbox}{max width=\textwidth}
\begin{tabular}{|P{0.18\linewidth}|P{0.20\linewidth}|P{0.20\linewidth}|P{0.20\linewidth}|P{0.20\linewidth}|P{0.20\linewidth}|}
\hline
\textbf{Dominio / Base de Datos} & \textbf{ACM Digital Library} & \textbf{IEEE Xplore} & \textbf{ScienceDirect} & \textbf{SpringerLink}  & \textbf{Taylor \& Francis} \\
\hline
\textbf{Educación AND VBC} 
& \tiny \texttt{(Title:(``Container-based virtualization'' OR ``Application virtualization'' OR ``Docker'' OR ``Lightweight Virtualization'') AND Title:(``Education'' OR ``Education System'' OR ``Education Development'' OR ``Higher Education'')) OR (Abstract:(``Container-based virtualization'' OR ``Application virtualization'' OR ``Docker'' OR ``Lightweight Virtualization'') AND Abstract:(``Education'' OR ``Education System'' OR ``Education Development'' OR ``Higher Education'')) OR (Keyword:(``Container-based virtualization'' OR ``Application virtualization'' OR ``Docker'' OR ``Lightweight Virtualization'') AND Keyword:(``Education'' OR ``Education System'' OR ``Education Development'' OR ``Higher Education''))} 
& \tiny \texttt{((``Abstract'': ``Container-based virtualization'' OR ``Abstract'': ``Application virtualization'' OR ``Abstract'': ``Docker'' OR ``Abstract'': ``Lightweight Virtualization'') AND (``Abstract'': ``Education'' OR ``Abstract'': ``Education System'' OR ``Abstract'': ``Education Development''  OR ``Abstract'': ``Higher Education'')) OR ((``Publication Title'': ``Container-based virtualization'' OR ``Publication Title'': ``Application virtualization'' OR ``Publication Title'': ``Docker'' OR ``Publication Title'': ``Lightweight Virtualization'') AND (``Publication Title'': ``Education'' OR ``Publication Title'': ``Education System'' OR ``Publication Title'': ``Education Development''  OR ``Publication Title'': ``Higher Education'')) OR ((``Author Keywords'': ``Container-based virtualization'' OR ``Author Keywords'': ``Application virtualization'' OR ``Author Keywords'': ``Docker'' OR ``Author Keywords'': ``Lightweight Virtualization'') AND (``Author Keywords'': ``Education'' OR ``Author Keywords'': ``Education System'' OR ``Author Keywords'': ``Education Development''  OR ``Author Keywords'': ``Higher Education''))} 
& \tiny \texttt{(``Container-based virtualization'' OR ``Application virtualization'' OR ``Docker'' OR ``Lightweight Virtualization'') AND (``Education'' OR ``Education System'' OR ``Education Development'' OR ``Higher Education'')} 
& \tiny \texttt{(title:(``Container-based virtualization'' OR ``Application virtualization'' OR ``Docker'' OR ``Lightweight Virtualization'') AND title:(``Education'' OR ``Education System'' OR ``Education Development'' OR ``Higher Education'')) OR (abstract:(``Container-based virtualization'' OR ``Application virtualization'' OR ``Docker'' OR ``Lightweight Virtualization'') AND abstract:(``Education'' OR ``Education System'' OR ``Education Development'' OR ``Higher Education'')) OR (keyword:(``Container-based virtualization'' OR ``Application virtualization'' OR ``Docker'' OR ``Lightweight Virtualization'') AND keyword:(``Education'' OR ``Education System'' OR ``Education Development'' OR ``Higher Education''))} 
& \tiny \texttt{(``Application virtualization'' OR ``Docker'' OR ``Lightweight Virtualization'' OR ``Docker Container'') AND (``Education System'' OR ``Education Sector'' OR ``Education Development'' OR ``Higher Education'')} \\
\hline

\hline
\textbf{Investigación AND VBC}
& \tiny \texttt{(Title:(``Container-based virtualization'' OR ``Application virtualization'' OR ``Docker'' OR ``Lightweight Virtualization'') AND Title:(``Research'' OR ``Research Group'' OR ``Research Proposal'')) OR (Abstract:(``Container-based virtualization'' OR ``Application virtualization'' OR ``Docker'' OR ``Lightweight Virtualization'') AND Abstract:(``Research'' OR ``Research Group'' OR ``Research Proposal'')) OR (Keyword:(``Container-based virtualization'' OR ``Application virtualization'' OR ``Docker'' OR ``Lightweight Virtualization'') AND Keyword:(``Research'' OR ``Research Group'' OR ``Research Proposal''))} 
& \tiny \texttt{((``Abstract'': ``Container-based virtualization'' OR ``Abstract'': ``Application virtualization'' OR ``Abstract'': ``Docker'' OR ``Abstract'': ``Lightweight Virtualization'') AND (``Abstract'': ``Research Group'' OR ``Abstract'': ``Research Proposal'')) OR ((``Publication Title'': ``Container-based virtualization'' OR ``Publication Title'': ``Application virtualization'' OR ``Publication Title'': ``Docker'' OR ``Publication Title'': ``Lightweight Virtualization'') AND (``Publication Title'': ``Research Group'' OR ``Publication Title'': ``Research Proposal'')) OR ((``Author Keywords'': ``Container-based virtualization'' OR ``Author Keywords'': ``Application virtualization'' OR ``Author Keywords'': ``Docker'' OR ``Author Keywords'': ``Lightweight Virtualization'') AND (``Author Keywords'': ``Research Group'' OR ``Author Keywords'': ``Research Proposal''))} 
& \tiny \texttt{(``Container-based virtualization'' OR ``Application virtualization'' OR ``Docker'' OR ``Lightweight Virtualization'') AND (``Research'' OR ``Research Group'' OR ``Research Proposal'')} 
& \tiny \texttt{(title:(``Container-based virtualization'' OR ``Application virtualization'' OR ``Docker'' OR ``Lightweight Virtualization'') AND title:(``Research'' OR ``Research Group'' OR ``Research Proposal'')) OR (abstract:(``Container-based virtualization'' OR ``Application virtualization'' OR ``Docker'' OR ``Lightweight Virtualization'') AND abstract:(``Research'' OR ``Research Group'' OR ``Research Proposal'')) OR (keyword:(``Container-based virtualization'' OR ``Application virtualization'' OR ``Docker'' OR ``Lightweight Virtualization'') AND keyword:(``Research'' OR ``Research Group'' OR ``Research Proposal''))} 
& \tiny \texttt{(``Application virtualization'' OR ``Docker'' OR ``Lightweight Virtualization'' OR ``Docker Container'') AND (``Specific Research Areas'' OR ``Research Group'' OR ``Research Proposal'' OR ``Research and Development'')} \\
\hline

\hline
\textbf{Industria AND VBC}
& \tiny \texttt{(Title:(``Container-based virtualization'' OR ``Application virtualization'' OR ``Docker'' OR ``Lightweight Virtualization'') AND Title:(``Industry'' OR ``IT Services'' OR ``Technology Infrastructure'' OR ``Cloud Computing'')) OR (Abstract:(``Container-based virtualization'' OR ``Application virtualization'' OR ``Docker'' OR ``Lightweight Virtualization'') AND Abstract:(``Industry'' OR ``IT Services'' OR ``Technology Infrastructure'' OR ``Cloud Computing'')) OR (Keyword:(``Container-based virtualization'' OR ``Application virtualization'' OR ``Docker'' OR ``Lightweight Virtualization'') AND Keyword:(``Industry'' OR ``IT Services'' OR ``Technology Infrastructure'' OR ``Cloud Computing''))} 
& \tiny \texttt{((``Abstract'': ``Container-based virtualization'' OR ``Abstract'': ``Application virtualization'' OR ``Abstract'': ``Docker'' OR ``Abstract'': ``Lightweight Virtualization'') AND (``Abstract'': ``Industry'' OR ``Abstract'': ``IT Services'' OR ``Abstract'': ``Technology Infrastructure'' OR ``Abstract'': ``Cloud Computing'')) OR ((``Publication Title'': ``Container-based virtualization'' OR ``Publication Title'': ``Application virtualization'' OR ``Publication Title'': ``Docker'' OR ``Publication Title'': ``Lightweight Virtualization'') AND (``Publication Title'': ``Industry'' OR ``Publication Title'': ``IT Services'' OR ``Publication Title'': ``Technology Infrastructure'' OR ``Publication Title'': ``Cloud Computing'')) OR ((``Author Keywords'': ``Container-based virtualization'' OR ``Author Keywords'': ``Application virtualization'' OR ``Author Keywords'': ``Docker'' OR ``Author Keywords'': ``Lightweight Virtualization'') AND (``Author Keywords'': ``Industry'' OR ``Author Keywords'': ``IT Services'' OR ``Author Keywords'': ``Technology Infrastructure'' OR ``Author Keywords'': ``Cloud Computing''))} 
& \tiny \texttt{(``Container-based virtualization'' OR ``Application virtualization'' OR ``Docker'' OR ``Lightweight Virtualization'') AND (``Industry'' OR ``IT Services'' OR ``Technology Infrastructure'' OR ``Cloud Computing'')} 
& \tiny \texttt{(title:(``Container-based virtualization'' OR ``Application virtualization'' OR ``Docker'' OR ``Lightweight Virtualization'') AND title:(``Industry'' OR ``IT Services'' OR ``Technology Infrastructure'' OR ``Cloud Computing'')) OR (abstract:(``Container-based virtualization'' OR ``Application virtualization'' OR ``Docker'' OR ``Lightweight Virtualization'') AND abstract:(``Industry'' OR ``IT Services'' OR ``Technology Infrastructure'' OR ``Cloud Computing'')) OR (keyword:(``Container-based virtualization'' OR ``Application virtualization'' OR ``Docker'' OR ``Lightweight Virtualization'') AND keyword:(``Industry'' OR ``IT Services'' OR ``Technology Infrastructure'' OR ``Cloud Computing''))} 
& \tiny \texttt{(``Application virtualization'' OR ``Docker'' OR ``Lightweight Virtualization'' OR ``Docker Container'') AND (``Industry'' OR ``IT Services'' OR ``Technology Infrastructure'' OR ``Cloud Computing'')} \\
\hline
\end{tabular}
\end{adjustbox}
\caption{Cadenas de búsqueda por dominio y base de datos}\label{tab:cadenas-busqueda}
\end{sidewaystable*}










% sub-subsetion for 4.2.3
\subsubsection{Estrategia de Búsqueda 2: Bola de Nieve (Snowballing)}
Quis ut deserunt in nulla aliquip exercitation. Voluptate non laborum do eu dolor mollit officia cupidatat do ea id id ullamco. Dolore velit anim est pariatur eiusmod occaecat duis labore reprehenderit nisi esse. Eu laboris cillum ullamco non velit veniam labore eiusmod laboris sint. Consequat officia aliquip velit officia do ex nulla cupidatat elit dolore deserunt sint.
\mbox{}\\

% sub=-subsection 4.2.4
\subsubsection{Resultados de la Búsqueda de Estudios}
\label{subsubsec:resultados-busqueda}
Nostrud enim magna culpa labore in culpa aliqua dolore ea amet sit magna exercitation sunt. Voluptate qui aliqua velit ipsum ullamco dolor ad velit cupidatat dolore sint. Nisi cillum dolore magna tempor minim ullamco anim quis ipsum consequat officia.
\mbox{}\\

%subsección 4.3
% subsection 4.3
\subsection{Stage 3: Quality Assessment}
According to~\cite{Ali-01}, the incorporation of a quality assessment is not mandatory in an SMS. However, such assessment could bring an SMS closer to a systematic review~\cite{Petersen-01}. Consequently, we sought to verify the relevance of the studies identified by the SMS objectives through this process. To carry out such assessment, the CVI (\textit{Content Value Index}), SCI (\textit{Study Citation Index}), and IRRQ (\textit{Index of Relationship to Research Questions}) indices defined during the planning stage were employed.

%sub-subsection 4.3.1
\subsubsection{Content Validity Assessment}
In this assessment, the content of the studies was analyzed to determine their value in the research context. For this purpose, the CVI index described in the planning stage was applied.

The rating of studies according to the CVI index was performed with the assistance of the SMS-Builder software. Subsequently, a frequency analysis was conducted to select the studies from the most significant quartile, that is, those with the highest CVI. This type of assessment is performed when identifying the \totalEtapaDos{} studies included in the SMS, whose results are presented in \hbox{Step 5: Study classification.}

%sub-subsection 4.3.2
\subsubsection{Index for Quality Assessment by Number of Citations}
The assessment of studies was performed by the team, taking into account the SCI index. We calculated this index with the support of the SMS-Builder software~\cite{sms-builder-repo} and citation data from Google Scholar. Subsequently, a frequency analysis was also conducted to select the studies from the most significant quartile, that is, those with the highest SCI.

%sub-subsection 4.3.3
\subsubsection{Index for the Evaluation of the Relationship of Studies to Research Questions}
This quality assessment uses the IRRQ index, described in the planning stage (Section~\ref{sec:planeacion}). In this regard, the selected studies were analyzed to establish their relationship with the classification topics.

Through this process, the SMS-Builder software~\cite{sms-builder-repo} was configured to establish the direct relationship of the studies with the research questions proposed in this SMS. As in the previous cases, a frequency analysis was also performed, where the quartile of most significant studies was selected, that is, those with the highest IRRQ.


%subsección 4.4
\input{secciones/04-metodo-revision/01-planeacion/01-objetivos-del-estudio.tex}

%subsección 4.5
\input{secciones/04-metodo-revision/01-planeacion/01-objetivos-del-estudio.tex}

%subsección 4.1
\input{secciones/04-metodo-revision/01-planeacion/01-objetivos-del-estudio.tex}


%Sección analisis y conclusiones
% Sección 5.1
\section{Análisis y discusión}
Lorem deserunt elit qui ad. Officia et excepteur eu ipsum culpa minim Lorem amet aliqua sit eiusmod. Do pariatur aute cillum non aute consectetur velit eu consectetur.

%Sección amenazas a la validez
% Sección 6
\section{Amenazas a la Validez}\label{sec:amenazas-validez}
Algunas de las principales limitaciones presentes en este estudio se relacionan con aspectos como los siguientes: 1) Sesgos en el proceso de selección de estudios. 2) Errores cometidos durante el proceso de clasificación de estudios. 3) Inexactitud en el proceso de extracción de datos. 4) Errores en la aplicación del protocolo de búsqueda.


% Subsección 6.1 
\subsection{Sesgo en la Selección de Estudios}
%!TODO En la parte 6 del siguiente parrafo, qué son las alertas? 
El sesgo en la selección de estudios fue mitigado mediante siete acciones.
Primero, se aplicaron de manera estricta los pasos del protocolo para la construcción de un estudio de mapeo sistemático, siguiendo las recomendaciones presentadas en la literatura \cite{Kitchenham2010792, budgen2008using}, e incorporando los modelos GQM y PICOC. Segundo, se utilizaron cinco bibliotecas digitales ampliamente reconocidas en el ámbito académico. Tercero, se incluyeron sinónimos de los términos principales de búsqueda con el fin de ampliar el alcance. Cuarto, se llevó a cabo la construcción iterativa de las cadenas de búsqueda mediante búsquedas piloto, las cuales permitieron realizar los ajustes necesarios para identificar la cantidad y calidad de los estudios. Quinto, se aplicó una estrategia híbrida de búsqueda, combinando la consulta en bases de datos con la técnica de snowballing. Estas estrategias posibilitaron ampliar el número de estudios relevantes en el SMS. Sexto, se generaron alertas para identificar estudios primarios utilizando las herramientas Endnote, Mendeley y Google Scholar. Séptimo, se realizó la validación de los estudios a partir de tres aspectos: a) evaluación de validez de contenido (CVI), b) evaluación de calidad según número de citaciones (SCI), y c) evaluación de la relación de los estudios con las preguntas de investigación (IRRQ). Por lo tanto, se considera que los posibles estudios no incluidos en este SMS tendrían un impacto reducido en los resultados. Cabe señalar que los índices CVI e IRRQ presentan como limitación el grado de subjetividad derivado de la opinión de los jueces al momento de la evaluación. Para reducir esta subjetividad, se promovió el trabajo colaborativo entre un número impar de evaluadores (mayor que uno).


% Subsección 6.2
\subsection{Errores en la Clasificación de Estudios}
Los estudios se clasificaron de acuerdo con las tópicos definidos en la etapa de planificación, las cuales guardan una relación estrecha con las preguntas de investigación.
Dichos tópicos corresponden a: a) Inteligencia artificial, b) Computación en la nube, c) Contenerización, d) Computación en malla, e) Computación de alto rendimiento (HPC), f) Java, g) Virtualización, h) Kubernetes, i) Redes, j) Paralelismo, k) Docker, l) Computación de alta productividad (HTC), m) Educación, n) Investigación, y o) Extensión.
Es importante señalar que algunos SMS fueron clasificados en múltiples tópicos debido a su cobertura temática intrínseca. Finalmente, la asociación entre estudios y tópicos se llevó a cabo mediante revisión por pares. Al igual que en el proceso de selección de estudios, con el fin de reducir sesgos en la clasificación de los SMS, se realizó un trabajo colaborativo entre revisores, incluyendo un número impar mayor que uno.

% Subsección 6.3 
\subsection{Inexactitud en el Proceso de Extracción de Datos}
Para la extracción de datos, se utilizó principalmente el software SMS-Builder \cite{sms-builder-repo}, el cual facilitó la extracción deductiva de datos de los SMS mediante su clasificación de acuerdo con la etapa de planificación. Asimismo, con el propósito de reducir posibles sesgos o errores en la extracción de información, se llevó a cabo una revisión por pares, conforme a las recomendaciones señaladas en \cite{Kitchenham2010792}.

% Subsección 6.4 
\subsection{Errores en la Aplicación del Protocolo de Búsqueda}
La aplicación de este protocolo se realizó mediante revisión por pares con el fin de reducir posibles errores en la ejecución. El primer evaluador siguió el protocolo de investigación y el segundo inspeccionó su trabajo. Para evitar el procesamiento manual de datos, se empleó el software SMS-Builder \cite{sms-builder-repo}, el cual contribuyó a disminuir la probabilidad de errores durante la aplicación del protocolo de búsqueda.


%Sección conclusiones
% Sección 7 
\section{Conclusiones}
Qui do adipisicing fugiat esse minim proident nisi sit incididunt excepteur. Aliquip ipsum ipsum amet Lorem ex exercitation consequat eu laborum do deserunt cupidatat. Nostrud officia esse ullamco incididunt dolor ipsum incididunt aute ea ipsum minim. Incididunt qui enim consectetur consectetur commodo nostrud voluptate. Ipsum proident et enim pariatur ullamco deserunt.



%Referencias
%% Sección 8
\section{Referencias}
Proident sit consectetur est irure Lorem aute eiusmod ipsum. Velit officia officia minim do proident ad. Do ex anim proident laborum cupidatat nulla ad.


% --- REFERENCIAS ---
% Estilo de la bibliografía
\bibliographystyle{IEEEtran}
% Archivo .bib donde se encuentran las referencias
\bibliography{resources/references.bib}
% -------------------

% ########################### FIN DE LAS SECCIONES ###########################

\end{document}




