% Sección 7 
\section{Conclusiones}\label{sec:conclusiones}


El presente artículo presenta un mapeo sistemático de estudios (SMS) que clasifican los trabajos existentes que hacen uso de HTCondor y sus contexto de ejecución, tambien llamados \textit{universos}. La investigación cubrió un período de 44 años, desde 1980 hasta 2024, buscando incluir tanto estudios actuales como fundacionales en la historia de esta tecnología. El SMS obtuvo 114 estudios primarios seleccionados (SPSs) a partir de una estrategia de búsqueda híbrida que combinó procedimientos automatizados para consultar cinco bibliotecas digitales y un procedimiento manual de "bola de nieve" para localizar estudios adicionales basados en citas y referencias.

En el SMS, definimos un esquema de clasificación de acuerdo con los tópicos determinados en la fase de planificación, los cuales responden a dos preguntas de investigación principales. Este esquema presenta la información del mapeo en tablas estadísticas y gráficos que ayudan a describir los datos y la relación existente con dichas preguntas de investigación. Con la realización del SMS, pudimos evidenciar la notable escasez de literatura que examine o clasifique sistemáticamente el uso específico de los universos de ejecución de HTCondor, a pesar de la amplia discusión de la herramienta en dominios como la computación en malla ("Grid Computing") y la computación de alta productividad (HTC). Esta situación confirma la carencia que motivó el presente estudio.

En consecuencia, este trabajo propone un esquema taxonómico que facilita la estructuración del cuerpo de documentación en esta área, ofreciendo una visión organizada del panorama actual. Esta estructura puede contribuir a la toma de decisiones informadas para la adopción e implementación de HTCondor y promover su uso óptimo en diversos escenarios.

Más allá de los resultados generales, el análisis de los estudios reveló tendencias subyacentes. Demostramos el papel protagónico de la computación en malla (``Grid Computing''), siendo el tópico más frecuente con un 34.21\% de los estudios, seguido de la computación de alto rendimiento y productividad (HPC y HTC). Además, se observó que el 100\% de los estudios se relacionan con la aplicación técnica de HTCondor (RQ1), mientras que solo un 24.56\% aborda simultáneamente su impacto en funciones universitarias como la investigación, docencia o extensión (RQ2), lo que indica una brecha en la literatura sobre su aplicación en contextos académicos más allá de la investigación pura.

Se espera que el presente estudio de mapeo sistemático sea de utilidad para futuros investigadores que desean explorar los distintos universos HTCodor, proporcionando una base sólida para comprender las tendencias actuales respecto de su aplicación. Además, se sugiere que futuras investigaciones podrían centrarse en evaluar el impacto de HTCondor en contextos educativos y de extensión teniendo en cuenta la limitada literatura encontrada al respecto, así como en explorar universos menos estudiados para diversificar y aportar a el conocimiento en este campo.

