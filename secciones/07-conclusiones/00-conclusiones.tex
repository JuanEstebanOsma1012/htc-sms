% Sección 7 
\section{Conclusiones}
Este estudio presenta un mapeo sistemático de la literatura (SMS) sobre tecnologías de virtualización basadas en contenedores relacionadas principalmente con la educación, la investigación y la extensión; por otra parte también se consideran los dominios de TI que utilizan estas tecnologías. El período de búsqueda abarca desde 2022 hasta 2024.\\
El SMS obtuvo 226 SPS que combinaron estrategias automatizadas para consultar las bases de datos académicas y búsquedas manuales para identificar estudios basados en citas y referencias relacionadas con el estudio.\\
En el SMS se definió un esquema de clasificación de acuerdo a determinados tópicos. Estos tópicos se definieron en la fase de planificación. Este esquema presenta un mapeo de la información en tablas y gráficos estadísticos, que ayudan a describir los datos y la relación con las preguntas de investigación.\\
Con la realización del mapeo sistemático, se evidenció una cada vez mayor proliferación de estudios relacionados con las tecnologías de virtualización basadas en contenedores, incluso en diferentes áreas y aplicaciones. Esta proliferación de estudios podría ser contraproducente para determinados \textit{stakeholders} a la hora de implementar arquitecturas con estas tecnologías. Lo anterior, podría llevar a una saturación de la literatura y dificultar el desarrollo, principalmente por el volumen de información disponible. \\
Por lo tanto, luego de completar el SMS, se propuso un esquema taxonómico basado en la clasificación desarrollada en este trabajo. Lo anterior tiene como objetivo facilitar la estructuración del cuerpo de conocimiento en esta área. Esta estructuración podría contribuir a la toma de decisiones, adopción de procesos o implementación de herramientas tecnologías con un enfoque en la virtualización basada en contenedores. \\
Más allá de los resultados obtenidos en este SMS, se analizaron los estudios explorando información subyacente o sus tendencias. Así, se mostró una creciente adopción de virtualización ligera en la implementación de soluciones, destacando la computación en la nube y los orquestadores como Kubernetes. Este liderazgo se ve fortalecido por la necesidad de las organizaciones de ofrecer redundancia, confiabilidad y escalabilidad en sus aplicaciones y servicios.\\
En cuanto al trabajo futuro, esperamos seguir comparando algunas de las herramientas tecnológicas relevantes para implementar entornos de prueba, analizando su comportamiento en diferentes escenarios.