% Sección 7 
\section{Conclusiones}\label{sec:conclusiones}
Este estudio presenta un mapeo sistemático de trabajos sobre los universos de HTCondor en dominios de computación distribuida y paralela, así como su vínculo con funciones sustantivas universitarias. La ventana de búsqueda considerada va de 1996 a 2024, con baja productividad antes de 2007 y un repunte posterior, lo que contextualiza la evolución del tema en el tiempo. 
Asimismo, se definieron dos preguntas de investigación: (RQ1) caracterización de universos y sus aplicaciones/estructuras en dominios técnicos; (RQ2) aportes de dichos universos a investigación, docencia, extensión e industria. 
La estrategia híbrida de búsqueda—automática en bases de datos y bola de nieve—produjo 114 estudios primarios seleccionados (SPS), de los cuales el 86.84\% provienen de bases de datos digitales y el 13.16\% de snowballing, lo que respalda la cobertura amplia del corpus. 
La totalidad de los SPS se relaciona con la RQ1; un 24.56\% contribuye simultáneamente a RQ1 y RQ2. No se hallaron estudios exclusivamente vinculados a RQ2, evidenciando una asimetría entre la caracterización técnico-funcional y la articulación con funciones sustantivas universitarias.

