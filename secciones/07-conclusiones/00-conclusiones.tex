% Section 7 
\section{Conclusions}\label{sec:conclusiones}

This article presents a Systematic Mapping Study (SMS) that classifies existing works that use HTCondor and its execution contexts, also known as \textit{universes}. The research covered a period of 44 years, from 1980 to 2024, aiming to include both current studies and foundational works in the history of this technology. The SMS retrieved 114 Selected Primary Studies (SPSs) through a hybrid search strategy that combined automated procedures to query five digital libraries and a manual “snowballing” procedure to identify additional studies based on citations and references.

In the SMS, a classification scheme according to the topics determined during the planning stage was defined, which address two main research questions. This scheme presents the mapping information in statistical tables and figures that help to describe the data and their relationship with the research questions. Through the SMS, a remarkable scarcity of literature that systematically examines or classifies the specific use of HTCondor execution universes was identified, despite the extensive discussion of the tool in domains such as Grid Computing and High Throughput Computing (HTC). This situation confirms the gap that motivated this study.

Consequently, tis work proposes a taxonomic scheme that facilitates the structuring of the body of documentation in this area, providing an organized view of the current landscape. This structure may contribute to informed decision-making for the adoption and implementation of HTCondor and foster its use in diverse scenarios.

Beyond the general results, the analysis of the studies revealed underlying trends. We demonstrated the leading role of Grid Computing, which emerged as the most frequent topic with 34.21\% of the studies, followed by High Performance Computing (HPC) and High Throughput Computing (HTC). Moreover, it was observed that 100\% of the studies are related to the technical application of HTCondor (RQ1), while only 24.56\% simultaneously address its impact on university functions such as research, teaching, or industry collaboration (RQ2), highlighting a gap in the literature concerning its application in academic contexts beyond pure research.

It is expected that this systematic mapping study will be useful for future researchers who wish to explore the different HTCondor universes, providing a solid foundation for understanding current trends in their application. Furthermore, future research could focus on assessing the impact of HTCondor in educational and industry collaboration contexts, given the limited literature found in this regard, as well as exploring less-studied universes to diversify and contribute to knowledge in this field.
