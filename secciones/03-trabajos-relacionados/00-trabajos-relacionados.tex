%sección III
\section{Related Work}\label{sec:trabajos-relacionados}
No prior studies were identified that share the specific objectives of this research. Furthermore, based on the findings of this work, it became evident that the literature lacks studies that examine HTCondor universes in particular, thereby confirming the need for this systematic mapping study.
However, the following related works were identified:
\begin{itemize}
	\item Erickson et al.~\cite{EricksonA-01} highlight the importance of technologies such as High Performance Computing (HPC) and High Throughput Computing (HTC) as enablers of computationally intensive research, emphasizing their potential in the environmental sciences.
	\item Tesser and Borin~\cite{KellerTesser2023} present a literature review discussing efforts to combine container-based virtualization technologies with distributed computing, with a particular focus on applications in the field of HPC.
	\item Raj et al.~\cite{RajRomanowski2020} explore the role of HPC as a subject of study for undergraduate students, underscoring the core competencies and skills required, as well as its applications in fields such as meteorology and biology.
	\item Thain, Tannenbaum, and Livny~\cite{Livny-Tannenbaum2005} describe the historical and contextual background that gave rise to the HTCondor software. The authors explain how HTCondor enables ordinary users to access substantial computational power through a distributed approach.
	\item Freyermuth, Wienemann, Bechtle, and Desch~\cite{Freyermuth2021a}. describe the commissioning and first operational experience gained with an HPC/HTC cluster that runs jobs in containers using Singularity and manages workloads with HTCondor.
\end{itemize}

Although these studies address aspects related to high-performance computing and the use of HTCondor in various domains, none of them focus specifically on the objectives of this research. In particular, the reviewed literature does not systematically examine HTCondor universes as a primary subject of study, nor does it provide a comprehensive analysis of their applications and characteristics in different research contexts. Consequently, a knowledge was is identified that justifies the present systematic mapping study, which seeks to consolidate and analyze in a structured manner the current state of knowledge on this specific topic.

