%sección III
\section{Trabajos Relacionados}\label{sec:trabajos-relacionados}

No se identificaron estudios previos que compartan los objetivos de
esta investigación. Adicionalmente, basandose en los resultados obtenidos
en el presente trabajo, se evidenció que la literatura carece de trabajos
que examinen específicamente los universos de HTCondor, confirmando
la necesidad del presente mapeo sistemático.

Sin embargo, se encontraron los siguientes trabajos relacionados:

\begin{itemize}[label=--]
	\item Erickson et al. en el 2018 \cite{EricksonA-01} describen la importancia de tecnologías como la computación
	      de alto rendimiento (HPC) y la computación de alta productividad (HTC) como medio para
	      realizar investigaciones computacionalmente intensivas, haciendo énfasis en el potencial
	      que tiene en las ciencias ambientales.

	\item Tesser R. y Borin E. en el 2022 \cite{KellerTesser2023} presentan una revisión de la literatura
	      donde se exponen los esfuerzos de combinar tecnologías como la virtualización basada
	      en contenedores y la computación distribuida, específicamente en el área de la computación
	      de alto rendimiento (HPC).

	      %Trabajo relacionado transitivamente con la pregunta de investigación 2.
	\item Raj et al. en el 2020 \cite{RajRomanowski2020} presentan los antecedentes de HPC como objeto de estudio para estudiantes de pregrado en las universidades,
	      resaltando las competencias y habilidades centrales que HPC requiere, así como su aplicación en distintos campos como la meteorología
	      y la biología.

	\item Thain D., Tannenbaum T. y Livny M. en el 2005 \cite{Livny-Tannenbaum2005} describen el contexto histórico y coyuntural que dio
	      origen al software HTCondor. Los autores describen en cómo HTCondor permite a los usuarios comunes
	      acceder a grandes cantidades de poder computacional a través de un enfoque distribuido.
\end{itemize}

A pesar de que los trabajos anteriormente listados abordan aspectos relacionados con la computación
de alto rendimiento y el uso de HTCondor en diversas áreas, ninguno de ellos se enfoca específicamente
en los objetivos planteados en la presente investigación. En particular, la literatura revisada
no examina de manera sistemática los universos de HTCondor como objeto de estudio principal,
ni proporciona un análisis exhaustivo de sus aplicaciones y características en diferentes
contextos de investigación. Por consiguiente, se identifica una brecha de conocimiento que
justifica la realización del presente estudio de mapeo sistemático, el cual permitirá consolidar y
 analizar de forma estructurada el estado actual del conocimiento sobre esta temática específica.