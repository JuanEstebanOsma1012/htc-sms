%sección III
\section{Trabajos relacionados}\label{sec:trabajos-relacionados}
Aunque se evidencian intentos de una estructuración del conocimiento en el área de las tecnologías de virtualización basadas en contenedores, no se encontró una revisión sistemática que permita identificar las tecnologías más relevantes según su caso de uso, ni una categorización en educación, investigación e industria. En este sentido, se presentan a continuación algunos trabajos relacionados que abordan aspectos relevantes de la virtualización basada en contenedores.

\begin{itemize}
    \item[\textendash] Sepúlveda et al. en 2022 \cite{SepulvedaRodriguez2022} presenta una revisión del estado del arte sobre modelos taxonómicos de tecnologías de virtualización, motivado por la dispersión de enfoques y la falta de uniformidad en la clasificación de estas tecnologías.
    \item[\textendash] Malhotra et al. en 2024 \cite{Malhotra2024} realizan una revisión sistemática de la literatura sobre tecnologías de virtualización, enfocándose en las fases del mantenimiento de contenedores, incluyendo la detección de imágenes de contenedores, la programación de contenedores, las medidas de seguridad de contenedores y la evaluación del rendimiento de contenedores.
    \item[\textendash] Kaiser et al. en 2023 \cite{Kaiser2023}  exploran el uso de tecnologías de contenedores en dispositivos de borde basados en arquitecturas ARM, evaluando su rendimiento, eficiencia y adecuación en escenarios de computación en el borde.
    \item[\textendash] Naydenov et al. en 2023 \cite{10094059}  realizan un mapeo sistemático de la literatura sobre arquitecturas, modelos y métodos de orquestación de contenedores en entornos de computación en la nube. El estudio tiene como objetivo identificar y clasificar los trabajos existentes, proponiendo un esquema de categorización que permita organizar el conocimiento en esta área.
    \item[\textendash] Kaiser et al. en 2022 \cite{Kaiser2022} realizan un análisis exhaustivo de tecnologías de contenedores compatibles con la arquitectura ARM, resaltando su consumo energético y alto rendimiento como factores clave para su adopción en contenerización. El estudio compara diversas tecnologías de contenedores, incluyendo orquestadores y entornos de ejecución, evaluando sus ventajas y desventajas frente a Docker.
    \item[\textendash] Bentaleb et al. en 2022 \cite{Bentaleb2022} presentan una revisión integral sobre tecnologías de virtualización y contenerización enfocadas en aplicaciones científicas intensivas. El estudio analiza taxonomías existentes de tecnologías de contenerización y propone una nueva que integra y complementa las previas.
\end{itemize}