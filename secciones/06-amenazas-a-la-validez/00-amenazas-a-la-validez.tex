% Section 6
\section{Threats to Validity}\label{sec:amenazas-validez}
Some of the main limitations present in this study are related to aspects such as the following: 1) Biases in the study selection process. 2) Errors made during the study classification process. 3) Inaccuracy in the data extraction process. 4) Errors in the application of the search protocol.

% Subsection 6.1 
\subsection{Bias in Study Selection}
%!TODO In part 6 of the following paragraph, what are the alerts? 
Bias in study selection was mitigated through seven actions.
First, the protocol steps for constructing a systematic mapping study were strictly applied, following the recommendations presented in the literature~\cite{Kitchenham2010792, budgen2008using}, and incorporating the GQM and PICOC models. Second, five digital libraries widely recognized in the academic field were used. Third, synonyms of the main search terms were included in order to broaden the scope. Fourth, iterative construction of search strings was carried out through pilot searches, which allowed for necessary adjustments to identify the quantity and quality of studies. Fifth, a hybrid search strategy was applied, combining database queries with the snowballing technique. These strategies enabled the expansion of the number of relevant studies in the SMS. Sixth, alerts were generated to identify primary studies using the Endnote, Mendeley, and Google Scholar tools. Seventh, study validation was performed based on three aspects: a) content validity assessment (CVI), b) quality assessment according to number of citations (SCI), and c) assessment of the relationship between studies and research questions (IRRQ). Therefore, it is considered that possible studies not included in this SMS would have a reduced impact on the results. It should be noted that the CVI and IRRQ indices present as a limitation the degree of subjectivity derived from the authors' opinion at the time of evaluation. To reduce this subjectivity, collaborative work was promoted among an odd number of evaluators.

% Subsection 6.2
\subsection{Errors in Study Classification}
Studies were classified according to the topics defined in the planning stage, which maintain a close relationship with the research questions.
These topics correspond to: a) Artificial intelligence, b) Cloud computing, c) Containerization, d) Grid computing, e) High Performance Computing (HPC), f) Java, g) Virtualization, h) Kubernetes, i) Networks, j) Parallelism, k) Docker, l) High Throughput Computing (HTC), m) Education, n) Research, and o) Industry.
It is important to note that some SPSs were classified into multiple topics due to their intrinsic thematic coverage. Finally, the association between studies and topics was carried out through peer review. As in the study selection process, in order to reduce biases in SMS classification, collaborative work was performed among reviewers, including an odd number greater than one.

% Subsection 6.3 
\subsection{Inaccuracy in the Data Extraction Process}
For data extraction, the SMS-Builder software~\cite{sms-builder-repo} was primarily used, which facilitated the deductive extraction of SMS data through their classification according to the planning stage. Likewise, in order to reduce possible biases or errors in information extraction, a peer review was carried out, in accordance with the recommendations indicated in~\cite{Kitchenham2010792}.

% Subsection 6.4 
\subsection{Errors in the Application of the Search Protocol}
The application of this protocol was performed through peer review in order to reduce possible errors in execution. The first evaluator followed the research protocol and the second inspected their work. To avoid manual data processing, the SMS-Builder software~\cite{sms-builder-repo} was employed, which contributed to reducing the probability of errors during the application of the search protocol.
