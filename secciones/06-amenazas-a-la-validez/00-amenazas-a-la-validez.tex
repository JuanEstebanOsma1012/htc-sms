% Sección 6
\section{Amenazas a la Validez}\label{sec:amenazas-validez}
Algunas de las principales limitaciones presentes en este estudio se relacionan con aspectos como los siguientes: 1) Sesgos en el proceso de selección de estudios. 2) Errores cometidos durante el proceso de clasificación de estudios. 3) Inexactitud en el proceso de extracción de datos. 4) Errores en la aplicación del protocolo de búsqueda.


% Subsección 6.1 
\subsection{Sesgo en la Selección de Estudios}
El sesgo en la selección de estudios fue mitigado mediante siete acciones.
Primero, se aplicaron de manera estricta los pasos del protocolo para la construcción de un estudio de mapeo sistemático, siguiendo las recomendaciones presentadas en la literatura \cite{Kitchenham2010792, budgen2008using}, e incorporando los modelos GQM y PICOC. Segundo, se utilizaron seis bibliotecas digitales ampliamente reconocidas en el ámbito académico. Tercero, se incluyeron sinónimos de los términos principales de búsqueda con el fin de ampliar el alcance. Cuarto, se llevó a cabo la construcción iterativa de las cadenas de búsqueda mediante búsquedas piloto, las cuales permitieron realizar los ajustes necesarios para identificar la cantidad y calidad de los estudios. Quinto, se aplicó una estratLegia híbrida de búsqueda, combinando la consulta en bases de datos con la técnica de snowballing. Estas estrategias posibilitaron ampliar el número de estudios relevantes en el SMS. Sexto, se generaron alertas para identificar estudios primarios utilizando las herramientas Endnote, Mendeley y Google Scholar. Séptimo, se realizó la validación de los estudios a partir de tres aspectos: a) evaluación de validez de contenido (CVI), b) evaluación de calidad según número de citaciones (SCI), y c) evaluación de la relación de los estudios con las preguntas de investigación (IRRQ). Por lo tanto, se considera que los posibles estudios no incluidos en este SMS tendrían un impacto reducido en los resultados. Cabe señalar que los índices CVI e IRRQ presentan como limitación el grado de subjetividad derivado de la opinión de los jueces al momento de la evaluación. Para reducir esta subjetividad, se promovió el trabajo colaborativo entre un número impar de evaluadores (mayor que uno).


% Subsección 6.2
\subsection{Errores en la Clasificación de Estudios}
%!TODO Aquí puse los topicos mencionados en la sección 4.1.1 -> ¿Son esos mismos los que van aquí o lo que 
% sobrevivieron luego del proceso SMS?
Los estudios se clasificaron de acuerdo con las tópicos definidos en la etapa de planificación, las cuales guardan una relación estrecha con las preguntas de investigación. 
Dichos tópicos corresponden a: a) Inteligencia artificial, b) Computación en la nube, c) Contenerización, d) Computación en malla, e) Computación de alto rendimiento (HPC), f) Java, g) Virtualización, h) Kubernetes, i) Redes, j) Paralelismo, k) Docker, l) Computación de alta productividad (HTC), m) Educación, n) Investigación, y o) Extensión.
Es importante señalar que algunos SMS fueron clasificados en múltiples tópicos debido a su cobertura temática intrínseca. Finalmente, la asociación entre estudios y tópicos se llevó a cabo mediante revisión por pares. Al igual que en el proceso de selección de estudios, con el fin de reducir sesgos en la clasificación de los SMS, se realizó un trabajo colaborativo entre revisores, incluyendo un número impar mayor que uno.

% Subsección 6.3 
\subsection{Inexactitud en el Proceso de Extracción de Datos}
Para la extracción de datos, se utilizó principalmente el software SMS-Builder \cite{sms-builder-repo}, el cual facilitó la extracción deductiva de datos de los SMS mediante su clasificación de acuerdo con la etapa de planificación. Asimismo, con el propósito de reducir posibles sesgos o errores en la extracción de información, se llevó a cabo una revisión por pares, conforme a las recomendaciones señaladas en \cite{Kitchenham2010792}.

% Subsección 6.4 
\subsection{Errores en la Aplicación del Protocolo de Búsqueda}
La aplicación de este protocolo se realizó mediante revisión por pares con el fin de reducir posibles errores en la ejecución. El primer evaluador siguió el protocolo de investigación y el segundo inspeccionó su trabajo. Para evitar el procesamiento manual de datos, se empleó el software SMS-Builder \cite{sms-builder-repo}, el cual contribuyó a disminuir la probabilidad de errores durante la aplicación del protocolo de búsqueda.
