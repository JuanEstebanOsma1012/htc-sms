% Sección 6
\section{Amenazas a la Validez}
Algunas de las principales limitaciones presentes en este estudio están relacionadas con aspectos como los siguientes: $1)$ Sesgo en el proceso de selección de estudios. $2)$ Errores cometidos durante el proceso de clasificación de los estudios. $3)$ Inexactitud en el proceso de extracción de datos. $4)$ Errores en la aplicación del protocolo de búsqueda.


% Subsección 6.1 
\subsection{Sesgo en la Selección de Estudios}
Se mitigó el el sesgo en la selección de estudios realizando siete acciones. En primer lugar, se aplicó estrictamente los pasos descritos en para construir un SMS, tal cual como se sugiere en los estudios~\cite{Runeson2009} y~\cite{Kitchenham2010}, incluyendo los modelos GQM y PICOC. En segundo lugar, se usó cinco bases de datos electrónicas ampliamente reconocidas en el ámbito académico. En tercer lugar, se incluyó sinónimos de los principales términos de búsqueda para dar amplitud según la aceptación que estos términos puedan tener en diferentes regiones, considerando que en este trabajo se incluyen estudios de diferentes países. En cuarto lugar, se realizó la construcción iterativa de las cadenas de búsqueda mediante búsquedas piloto. Estas búsquedas permitieron realizar los ajustes necesarios para identificar la cantidad y la calidad de los estudios. En quinto lugar, se usó una estrategia de búsqueda híbrida entre la investigación en bases de datos y el método de bola de nieve. Estas estrategias ampliaron el número de estudios relevantes en el SMS. En sexto lugar, se crearon alertas para identificar estudios primarios mediante herramientas como Endnote, Mendeley y Google Scholar. En séptimo lugar, se cuantificaron los estudios a través de tres aspectos: $a)$ evaluación de la validez de contenido (CVI), $b)$ evaluación de la calidad según el número de citas (SCI), y $c)$ evaluación de la relación de los estudios con las preguntas de investigación (IRRQ). Por lo tanto, se considera que los estudios no incluidos en este SMS tendrían un impacto bajo en los resultados. Los índices CVI e IRRQ presentan una deficiencia debido al grado de subjetividad derivado de la opinión de cada investigador al evaluar los estudios. Para reducir esta subjetividad, se utilizó el trabajo colaborativo entre un número impar de evaluadores (más de uno).

% Subsección 6.2
\subsection{Errores en la Clasificación de Estudios}
Se clasificaron los estudios según los temas definidos en la etapa de planificación. Estos temas tienen una fuerte relación con las preguntas de investigación. Corresponden a: $a)$ marco, b) modelo, c) metodología, d) proceso, e) buenas prácticas, f) herramientas tecnológicas y g) estudio de caso o proyecto. Es fundamental aclarar que se  algunos estudios de caso en diferentes temas debido a su cobertura temática intrínseca. Finalmente, realizamos la comparación entre estudios y temas mediante revisión por pares. Al igual que en el proceso de selección de estudios, para reducir el sesgo en la clasificación de los estudios de caso con los temas, se realizó un trabajo colaborativo entre los revisores, incluyendo un número impar mayor que uno.
    
% Subsección 6.3 
\subsection{Inexactitud en el Proceso de Extracción de Datos}
Magna incididunt cupidatat cupidatat enim eu ea commodo mollit labore aute id eiusmod. Ullamco cillum occaecat aute ullamco nisi. Est nulla adipisicing dolore sunt officia aliquip quis ut. Mollit veniam enim commodo labore proident sunt commodo quis consectetur Lorem. Officia veniam qui sint nisi consectetur id et mollit cillum non tempor magna quis reprehenderit. Veniam consectetur veniam officia esse officia pariatur ullamco nisi occaecat occaecat.

% Subsección 6.4 
\subsection{Errores en la Aplicación del Protocolo de Búsqueda}
Anim velit tempor cupidatat in excepteur pariatur labore proident laborum ad officia ipsum aute. Et excepteur in aute eiusmod velit labore nulla amet proident. Aliquip officia et et sunt voluptate ut commodo. Dolore magna culpa id ullamco mollit ut nulla adipisicing dolor.

