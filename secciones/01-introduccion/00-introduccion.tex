%sección I
\section{Introducción}
\IEEEPARstart{E}{l} objetivo de la computación científica es la resolución 
de problemas. La computadora resulta necesaria para este propósito debido a 
que algunos problemas del mundo real frecuentemente presentan un nivel de dificultad 
o complejidad que excede las capacidades de analítica o resolución humana, sin embargo,
estos pueden ser abordados efectivamente mediante el uso 
de recursos computacionales~\cite{landau01}. 


No obstante, no todos los problemas cientificos son
manejables para una sola computadora. Exiten problemas cuya ejecución en una sola maquina resulta 
inviable debido a factores como su naturaleza o a el tamaño de su conjunto de datos. Es por esto 
que los investigadores usan herramientas de computación de alta productividad 
(\textit{Hight Throughput Computing} o HTC por sus siglas en inglés), las cuales tienen como 
propósito el maximizar la cantidad de resultados producidos durante un periodo 
largo de tiempo~\cite{juve-01}.

Es en este contexto nace HTCondor, un sistema especializado de gestión de 
cargas de trabajo para tareas de cómputo intensivo~\cite{chang-01}.

