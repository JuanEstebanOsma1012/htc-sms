%sección I
\section{Introducción}
\IEEEPARstart{E}{l} objetivo de la computación científica es la resolución
de problemas. La computadora resulta necesaria para este propósito debido a
que algunos problemas del mundo real frecuentemente presentan un nivel de dificultad
o complejidad que excede las capacidades de analítica o resolución humana, sin embargo,
estos pueden ser abordados efectivamente mediante el uso
de recursos computacionales~\cite{landau01}.


No obstante, no todos los problemas cientificos son
manejables para una sola computadora. Exiten problemas cuya ejecución en una sola maquina resulta
inviable debido a factores como su naturaleza o a el tamaño de su conjunto de datos. Es por esto
que los investigadores usan herramientas de computación de alta productividad
(\textit{Hight Throughput Computing} o HTC por sus siglas en inglés), las cuales tienen como
propósito el maximizar la cantidad de resultados producidos durante un periodo
largo de tiempo~\cite{juve-01}.

En este contexto surge HTCondor, un sistema creado por la
Universidad Wisconsin–Madison, especializado en la gestión de cargas
de trabajo y diseñado específicamente para tareas de cómputo intensivo~\cite{chang-01, htcondor-description}.
HTCondor permite a los científicos ejecutar tareas en un clúster dedicado, mientras que el
software se encarga automáticamente de distribuir el trabajo entre los diferentes
nodos de ejecución. Esta distribución se realiza siguiendo políticas establecidas
tanto por los usuarios propietarios de los nodos ejecutores como por los
propios científicos que solicitan los recursos~\cite{htcondor-description}.

Dichos trabajos computacionales vienen en la forma de lenguajes de programación
o modos de ejecución los cuales HTCondor llama \textit{contextos de ejecución}
o \textit{universos} 