%sección I
\section{Introduction}
\label{sec:introduccion}
Scientific computing addresses complex problems that exceed the scope of manual analytical approaches~\cite{landau01}. Computers are indispensable in this context, as many real-world problems involve levels of difficulty or data scale that make purely analytical approaches infeasible; nevertheless, making use of computational resources allows these problems to be effectively addressed~\cite{landau01}.

However, many scientific problems cannot be effectively addressed by a single computer due to their computational complexity. Certain problems are computationally infeasible to solve on a single machine due to factors such as their inherent complexity or the size of their input datasets. For this reason, researchers employ distributed computing tools to produce results in a reasonable time. Some of these tools belong to the realm of High Throughput Computing (HTC), a paradigm whose goal is to maximize the volume of results produced over extended periods of time~\cite{juve-01} and which has garnered interest in educational contexts~\cite{Senol-01}.

Within this domain, HTCondor, developed at the University of Wisconsin–Madison, has emerged as a workload management system designed specifically for compute-intensive applications~\cite{chang-01, htcondor-description}. HTCondor allows users to submit tasks to a pool, where the system autonomously manages resource allocation, scheduling, and distribution across nodes. Its scheduling mechanism follows a bidirectional policy: both resource owners and job submitters may define conditions and preferences that influence how and where tasks are executed~\cite{htcondor-description}.

HTCondor organizes computational jobs into distinct execution environments, known as universes. At the time of writing, the available universes include: vanilla, grid, java, scheduler, local, parallel, vm, container, and docker. This diversity reflects the system’s adaptability, ranging from traditional computing tasks to virtualized and containerized environments.

Despite this versatility, the literature lacks a systematic classification that clarifies the contexts in which HTCondor universes are applied and their broader impact. To address this gap, this paper presents a systematic mapping study (SMS) with two objectives: first, to classify works across technological domains such as distributed and parallel computing, software development, virtualization, containerization, and networking, among others; and second, to identify and categorize works that leverage HTCondor universes as tools supporting academic functions (AF) such as research, teaching, and industry collaboration.

The remainder of this document is structured as follows:
Section~\ref{sec:motivacion} outlines the motivation for this work.
Section~\ref{sec:trabajos-relacionados} reviews related studies.
Section~\ref{sec:metodo-revision} describes the method employed to conduct the SMS.
Section~\ref{sec:analisis-discusion} presents the analysis and discussion of the study.
Section~\ref{sec:amenazas-validez} discusses threats to validity, and finally,
Section~\ref{sec:conclusiones} provides the conclusions.