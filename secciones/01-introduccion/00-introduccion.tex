%sección I
\section{Introducción}
\IEEEPARstart{L}{a} computación en la nube se ha consolidado como una de las tendencias más influyentes en el ámbito tecnológico contemporáneo, propendiendo por soluciones escalables y resilientes. Esta forma de ofrecer servicios de computo se halla soportada por diversas tecnologías, entre las que destacan la virtualización completa y los contenedores. La virtualización completa permite la creación de entornos aislados y robustos, mientras que los contenedores proporcionan una forma ligera y portátil de empaquetar aplicaciones y sus dependencias.\\
Con el crecimiento de la computación en la nube, las tecnologías de virtualización basadas en contenedores han ganado una relevancia significativa. Estas tecnologías permiten a las organizaciones desplegar, gestionar y escalar aplicaciones de manera flexible, simplificando el uso de recursos y facilitando la integración continua y el desarrollo ágil. Herramientas como Docker y Kubernetes han emergido como líderes en este campo, ofreciendo soluciones robustas para la gestión de contenedores y orquestación de aplicaciones en entornos distribuidos.\\
A lo anterior se añade que Docker es considerado un estándar de \textit{facto} en la industria para la creación y gestión de contenedores; no obstante, puede ser que sus características no se ajusten a todos los casos de uso. Es debido a lo anterior que surge la necesidad de explorar la literatura con el fin de identificar alternativas y complementos en el ecosistema de contenedores. Para delimitar la investigación y explorar los posibles casos de uso se definieron dominios de TI que abarcan desde el desarrollo de software, infraestructura de TI, HPC, seguridad, entre otros e interpretar su relación con las tecnologías de virtualización ligera.\\
Por otra parte y desde una perspectiva académica, se ha extrapolado el uso de estas tecnologías a ámbitos como la educación, la investigación y la extensión. En estos contextos, las tecnologías de virtualización basadas en contenedores podrían ofrecer beneficios significativos, tales como la creación de entornos de aprendizaje personalizados, la facilitación de la colaboración entre investigadores y la provisión de recursos computacionales accesibles para proyectos de industria.\\