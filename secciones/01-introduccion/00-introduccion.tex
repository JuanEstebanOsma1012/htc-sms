%sección I
\section{Introducción}
\IEEEPARstart{E}{l} objetivo de la computación científica es la resolución
de problemas. La computadora resulta necesaria para este propósito debido a
que algunos problemas del mundo real frecuentemente presentan un nivel de dificultad
o complejidad que excede las capacidades de analítica o resolución humana, sin embargo,
estos pueden ser abordados efectivamente mediante el uso
de recursos computacionales~\cite{landau01}.


No obstante, no todos los problemas científicos son
manejables para una sola computadora. Existen problemas cuya ejecución en una sola maquina resulta
inviable debido a factores como su naturaleza o a el tamaño de su conjunto de datos. Es por esto
que los investigadores usan herramientas de computación de alta productividad
(\textit{Hight Throughput Computing} o HTC por sus siglas en inglés), las cuales tienen como
propósito el maximizar la cantidad de resultados producidos durante un periodo
largo de tiempo~\cite{juve-01},
%El propósito de la siguiente linea es el de servir de preambulo para el objetivo 2 del SMS. 
tecnologías propias de la computación distribuida, disciplina que también
ha encontrado interés en el ámbito educativo~\cite{Senol-01}.

En este contexto surge HTCondor, un sistema creado por la
Universidad Wisconsin–Madison, especializado en la gestión de cargas
de trabajo y diseñado específicamente para tareas de cómputo intensivo~\cite{chang-01, htcondor-description}.
HTCondor permite a los usuarios enviar tareas computacionales a un clúster,
donde el sistema gestiona de forma autónoma la asignación, planificación y distribución del
trabajo entre los nodos disponibles. El mecanismo de planificación opera bajo un
modelo de políticas bidireccional: tanto los propietarios de los recursos computacionales
como los usuarios solicitantes pueden establecer criterios y preferencias
que determinan dónde y bajo qué condiciones se ejecutarán las tareas~\cite{htcondor-description}.
Dichos trabajos computacionales vienen en la forma de lenguajes de programación
o \textit{contextos de ejecución} los cuales HTCondor llama
\textit{universos}. Hasta la fecha en la que se escribe el presente artículo
los universos HTCondor disponibles son los siguientes: \textit{vanilla, grid, java, scheduler,
	local, parallel, vm, container y docker}.


La diversidad de universos disponibles refleja la amplia gama de aplicaciones
que HTCondor puede soportar, desde computación tradicional hasta entornos
virtualizados y contenedorizados. No obstante, la literatura científica carece
de una clasificación sistemática que permita comprender cómo estos universos
se aplican en diferentes contextos y cuál es su impacto. En consecuencia,
el presente documento exponemos un estudio de mapeo sistemático que busca,
en primer lugar, clasificar trabajos relacionados con diversos dominios tecnológicos
como lo son la computación distribuída y paralela, el desarrollo de software,
virtualización, contenerización, la redes de computadora, entre otros. En segundo lugar,
se busca identificar y categorizar trabajos vinculados con los universos de
HTCondor como herramienta para fortalecer funciones esenciales universitarias
como: investigación, docencia, extensión e industria.

El resto del documento se estructura de la siguiente manera:
la Sección~\ref{sec:motivacion} indica la motivación para este trabajo.
La Sección~\ref{sec:trabajos-relacionados} presenta trabajos relacionados.
La Sección~\ref{sec:metodo-revision} describe el método utilizado para llevar a cabo el SMS.
La Sección~\ref{sec:analisis-discusion} contiene el análisis y discusión del trabajo realizado.
La Sección~\ref{sec:amenazas-validez} discute las amenazas a la validez, y finalmente,
la Sección~\ref{sec:conclusiones} presenta las conclusiones.