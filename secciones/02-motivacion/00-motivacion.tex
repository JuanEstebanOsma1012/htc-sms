%sección II
\section{Motivación}
La computación en la nube ha introducido un abanico de soluciones ampliamente adoptadas; sin embargo, aún persiste una fragmentación en la literatura y una falta de sistematización, principalmente debido al alto volumen de información, lo que dificulta identificar patrones claros de uso, beneficios y limitaciones en distintos dominios de aplicación.

Este trabajo se motiva en la necesidad de obtener una visión estructurada que permita reconocer cómo estas tecnologías están siendo adoptadas en ámbitos industriales y académicos, así como los factores que condicionan su uso. Los resultados esperados buscan ofrecer un mapa comprensivo de tendencias, enfoques y prácticas emergentes que sirva de guía para investigadores y profesionales en la toma de decisiones tecnológicas.

En particular, se pretende identificar estudios relacionados con la virtualización basada en contenedores \textbf{(VBC)} en diferentes dominios de TI, así como aquellos vinculados directamente con el ámbito académico, especialmente en las áreas de educación, investigación y extensión. Esta clasificación busca aportar claridad sobre el panorama actual y, al mismo tiempo, fomentar la exploración de nuevas líneas de investigación y aplicaciones prácticas en múltiples escenarios.