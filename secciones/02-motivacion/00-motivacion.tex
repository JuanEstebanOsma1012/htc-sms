%sección II
\section{Motivación}\label{sec:motivacion}


A pesar de la adopción de HTCondor en diversos dominios científicos y tecnológicos,
existe una notable escasez a en la literatura respecto al uso específico de sus universos de ejecución.

La ausencia de una clasificación sistemática de los trabajos relacionados
con los universos de HTCondor impide el desarrollo de decisiones informadas para su implementación y
limita la transferencia efectiva de conocimiento entre diferentes comunidades de usuarios.
Además, la literatura actual no ofrece una perspectiva consolidada sobre cómo estos
universos contribuyen específicamente al fortalecimiento de las actividades de investigación,
docencia, extensión e industria en el contexto universitario. Esta carencia de sistematización
del conocimiento existente motivó la realización del presente estudio de mapeo sistemático,
con el propósito de proporcionar una visión estructurada y comprehensiva que facilite la toma
de decisiones informadas y promueva el uso óptimo de HTCondor en diversos escenarios de aplicación.