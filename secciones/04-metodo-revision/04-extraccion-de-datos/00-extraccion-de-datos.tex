% subseccion 4.4
\subsection{Etapa 4: Extracción de Datos}\label{sec:extraccion-de-datos}

Una vez aplicados los procedimientos de búsqueda y selección de estudios, así como la evaluación de su calidad, se seleccionaron \totalEtapaDos{} estudios primarios. Estos estudios se etiquetaron con el prefijo SPS (Selected Primary Study) seguido de un número. La lista completa de estos estudios se encuentra en la Tabla \ref{table:selected_primary_studies}.

A partir de los \totalEtapaDos{} estudios primarios seleccionados (SPS), se diseñaron y completaron los formularios para obtener datos relevantes para la SMS. La información contenida en los SPSs se relacionó con las preguntas de investigación a través de los temas definidos en la fase de planificación y consignados mediante los modelos GQM y PICOC. Además, se extrajeron metadatos como palabras clave para identificar aspectos subyacentes de los SPSs.


% -------- Tabla : Estudios Primarios Seleccionados (SPSs) ------------
\begin{table*}[htbp]
	\centering
	\caption{Los \totalStudies{} estudios primarios seleccionados (SPSs)}
	\label{table:selected_primary_studies}
	\renewcommand{\arraystretch}{1.2}
	\setlength{\tabcolsep}{4pt} % Reduce space between columns
	\begin{tabular*}{\textwidth}{l @{\extracolsep{\fill}} r l @{\extracolsep{\fill}} r l @{\extracolsep{\fill}} r l @{\extracolsep{\fill}} r l @{\extracolsep{\fill}} r}
		\toprule
		\textbf{ID} & \textbf{Ref} & \textbf{ID} & \textbf{Ref} & \textbf{ID} & \textbf{Ref} & \textbf{ID} & \textbf{Ref} & \textbf{ID} & \textbf{Ref} \\
		\midrule
		SPS001      & [1]          & SPS002      & [2]          & SPS003      & [3]          & SPS004      & [4]          & SPS005      & [5]          \\
		SPS006      & [6]          & SPS007      & [7]          & SPS008      & [8]          & SPS009      & [9]          & SPS010      & [10]         \\
		SPS011      & [11]         & SPS012      & [12]         & SPS013      & [13]         & SPS014      & [14]         & SPS015      & [15]         \\
		SPS016      & [16]         & SPS017      & [17]         & SPS018      & [18]         & SPS019      & [19]         & SPS020      & [20]         \\
		SPS021      & [21]         & SPS022      & [22]         & SPS023      & [23]         & SPS024      & [24]         & SPS025      & [25]         \\
		SPS026      & [26]         & SPS027      & [27]         & SPS028      & [28]         & SPS029      & [29]         & SPS030      & [30]         \\
		SPS031      & [31]         & SPS032      & [32]         & SPS033      & [33]         & SPS034      & [34]         & SPS035      & [35]         \\
		SPS036      & [36]         & SPS037      & [37]         & SPS038      & [38]         & SPS039      & [39]         & SPS040      & [40]         \\
		SPS041      & [41]         & SPS042      & [42]         & SPS043      & [43]         & SPS044      & [44]         & SPS045      & [45]         \\
		SPS046      & [46]         & SPS047      & [47]         & SPS048      & [48]         & SPS049      & [49]         & SPS050      & [50]         \\
		SPS051      & [51]         & SPS052      & [52]         & SPS053      & [53]         & SPS054      & [54]         & SPS055      & [55]         \\
		SPS056      & [56]         & SPS057      & [57]         & SPS058      & [58]         & SPS059      & [59]         & SPS060      & [60]         \\
		SPS061      & [61]         & SPS062      & [62]         & SPS063      & [63]         & SPS064      & [64]         & SPS065      & [65]         \\
		SPS066      & [66]         & SPS067      & [67]         & SPS068      & [68]         & SPS069      & [69]         & SPS070      & [70]         \\
		SPS071      & [71]         & SPS072      & [72]         & SPS073      & [73]         & SPS074      & [74]         & SPS075      & [75]         \\
		SPS076      & [76]         & SPS077      & [77]         & SPS078      & [78]         & SPS079      & [79]         & SPS080      & [80]         \\
		SPS081      & [81]         & SPS082      & [82]         & SPS083      & [83]         & SPS084      & [84]         & SPS085      & [85]         \\
		SPS086      & [86]         & SPS087      & [87]         & SPS088      & [88]         & SPS089      & [89]         & SPS090      & [90]         \\
		SPS091      & [91]         & SPS092      & [92]         & SPS093      & [93]         & SPS094      & [94]         & SPS095      & [95]         \\
		SPS096      & [96]         & SPS097      & [97]         & SPS098      & [98]         & SPS099      & [99]         & SPS100      & [100]        \\
		SPS101      & [101]        & SPS102      & [102]        & SPS103      & [103]        & SPS104      & [104]        & SPS105      & [105]        \\
		SPS106      & [106]        & SPS107      & [107]        & SPS108      & [108]        & SPS109      & [109]        & SPS110      & [110]        \\
		SPS111      & [111]        & SPS112      & [112]        & SPS113      & [113]        & SPS114      & [114]        &             &              \\
		\bottomrule
	\end{tabular*}
\end{table*}
% --------------------------------------------------------------


El proceso fue respaldado por diversas herramientas informáticas, como hojas de cálculo, software especializado en la gestión de referencias bibliográficas (por ejemplo, EndNote y Mendeley), y SMS-Builder \cite{sms-builder-repo} para la construcción del análisis de datos. Toda la información de este SMS ha sido publicada en el sitio \cite{sms-builder-own-container}.

Los interesados pueden obtener un contenedor Docker \textcolor{red}{(!TODO Colocar el link del Dockerhub donde se encuentra alojado el contenedor con todo el SMS)} que incluye una versión funcional de SMS-Builder con todos los datos utilizados en este estudio. Se invita a los lectores a interactuar con esta herramienta tecnológica para obtener información complementaria a la que se presenta en este artículo.
