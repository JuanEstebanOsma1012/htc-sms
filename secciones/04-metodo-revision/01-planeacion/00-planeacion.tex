% subseccion 4.1  
\subsection{Fase 1: Planeación}
En esta etapa se estableció el propósito general para el SMS y se definieron los objetivos del estudio, preguntas de investigación, métricas, tópicos de clasificación, criterios de inclusión y exclusión y criterios de calidad. Ver Figura 2.
Para los componentes ``Objetivos del estudio'', ``Preguntas de investigación'' y ``Métricas'', de la etapa de planificación, se aplicó el modelo {\itshape Goal-Question-Metric} (GQM por sus siglas en inglés) [Sepulveda's article citations 20 and 21]. Estos componentes consideran el nivel conceptual, operacional y cuantitativo, respectivamente según Sepulveda et al [Referencia del articulo de Sepúlveda].

% 4.1.1
%sub-subseccion - objetivos del estudio
\subsubsection{Objetivos del estudio}
Teniendo en cuenta los aspectos descritos en la sección de motivación, se han definido dos objetivos para el estudio, detallados en la Tabla \ref{table:Goals}.

\begin{table}[htbp]
	
	\centering
	\renewcommand{\arraystretch}{1.7}  % Espaciado entre filas
	\renewcommand{\tablename}{Tabla}  % <- Cambia "Cuadro" por "Tabla"
	\setlength{\tabcolsep}{3pt}      % Espaciado horizontal en celdas
	\vspace{10pt}                     % Espacio arriba de la tabla (opcional)
	\begin{tabular}{|>{\arraybackslash}m{1cm}|>{\arraybackslash}m{7cm}|}
		\hline	
		\textbf{Objetivo} & \textbf{Descripción} \\
		\hline
		G1 & Clasificar trabajos relacionados con los universos de HTCondor según su aplicación e impacto en los dominios de computación distribuida y paralela, HTC, desarrollo de Software, virtualización y microservicios, redes de computadoras, infraestructura computacional, inteligencia artificial, análisis de datos y pensamiento computacional, entre otros.\\
		G2 & Identificar y categorizar trabajos vinculados con los universos de HTCondor como herramienta para fortalecer funciones esenciales universitarias como: investigación, docencia, extensión e industria.\\
		\hline
	\end{tabular}
	\vspace{6pt}  % Espacio entre tabla y caption
	\caption{Objetivos del SMS.}
	\label{table:Goals}
	
\end{table}

% 4.1.2
%sub-subseccion - pregunta de investigación
\subsubsection{Pregunta de investigación}
Para la construcción de las preguntas de investigación (RQ por sus siglas en inglés) se usó el modelo PICOC, que nos permite establecer los aspectos "Población", "Intervención", "Comparación", "Resultados" y "Contexto". Todo esto para situar el trabajo en un contexto adecuado y propender por la entrega de valor. Ver Tabla \ref{table:PICOC}.

\begin{table}[htbp]
	
	\centering
	\renewcommand{\arraystretch}{1.7}  % Espaciado entre filas
	\renewcommand{\tablename}{Tabla}  % <- Cambia "Cuadro" por "Tabla"
	\setlength{\tabcolsep}{3pt}      % Espaciado horizontal en celdas
	\vspace{10pt}                     % Espacio arriba de la tabla (opcional)
	\begin{tabular}{|>{\arraybackslash}m{1.7cm}|>{\arraybackslash}m{6.3cm}|}
		\hline	
		\textbf{Componente} & \textbf{Descripción} \\
		\hline
		Población & Trabajos relacionados con los universos de HTCondor según su aplicación e impacto en los dominios de computación distribuida y paralela, HTC, desarrollo de Software, virtualización y microservicios, redes de computadoras, infraestructura computacional, inteligencia artificial, análisis de datos, pensamiento computacional, entre otros. Que potencian las funciones sustantivas universitarias de investigación, docencia y extensión.\\
		Intervención & Identificación y clasificación de un conjunto de trabajos relacionados con los universos de HTCondor según su aplicación e impacto en los dominios de computación distribuida y paralela, HTC, desarrollo de Software, virtualización y microservicios, redes de computadoras, infraestructura computacional, inteligencia artificial, análisis de datos, pensamiento computacional, entre otros. Que potencian las funciones sustantivas universitarias de investigación, docencia y extensión.\\
		Comparación & Casos de proyecto documentados; Cumplimiento de criterios de inclusión y exclusión;
		Aparición en bases de datos seleccionadas.\\
		Salidas & Taxonomía que organiza los trabajos relacionados con los universos de HTCondor según su aplicación e impacto en los dominios de computación distribuida y paralela, HTC, desarrollo de Software, virtualización y microservicios, redes de computadoras, infraestructura computacional, inteligencia artificial, análisis de datos, pensamiento computacional, entre otros. Que potencian las funciones sustantivas universitarias de investigación, docencia y extensión.\\
		Contexto & Universos HTCondor en dominios de computación distribuida y paralela, HTC, desarrollo de Software, virtualización y microservicios, redes de computadoras, infraestructura computacional, inteligencia artificial, análisis de datos, pensamiento computacional, entre otros. Que potencian las funciones sustantivas universitarias de investigación, docencia y extensión.\\
		\hline
	\end{tabular}
	\vspace{6pt}  % Espacio entre tabla y caption
	\caption{Objetivos del SMS.}
	\label{table:PICOC}
	
\end{table}

% 4.1.3
%sub-subseccion - metricas
\subsubsection{Métricas}
\mbox{}\\

% 4.1.4
%sub-subseccion - topicos de investigación
\subsubsection{Tópicos de investigación}
\mbox{}\\

% 4.1.5
%sub-subseccion - criterios de inclusión y exclusión
\subsubsection{Criterios de inclusión y exclusión}
\mbox{}\\

% 4.1.6
% sub-subseccion - Criterios de calidad
\subsubsection{Criterios de calidad}
\mbox{}\\
