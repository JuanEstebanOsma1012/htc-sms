% subseccion 4.1  
\subsection{Stage 1: Planning}\label{sec:planeacion}

At this stage, we established the general purpose of the research and defined goals, research questions, metrics, classification topics, inclusion/exclusion criteria,and quality criteria. See Fig. \ref{fig:PlanningStageOverview}.

For the components ``Study Objectives'', ``Research Questions'', and ``Metrics'' of the planning stage, the {\itshape Goal-Question-Metric} (GQM) model \cite{basili1992software, caldiera1994goal} was applied. These components consider the conceptual, operational, and quantitative levels, respectively, according to \cite{Sepúlveda202141}. % !TODO Es apropiado colocar esta última cita?  

\begin{figure}[htbp]
	\centering
	\includegraphics[scale=0.4]{resources/figures/sms-Etapa-1 overview.drawio.png}
	\caption{Components of the planning stage}
	\label{fig:PlanningStageOverview}
\end{figure}

% 4.1.1
%sub-subsection - study objectives
\subsubsection{Study Objectives}
Taking into account the aspects described in the motivation section, two objectives were defined for the study, detailed in Table \ref{table:Goals}.

\begin{table}[htbp]
	\centering
	\caption{Objectives of the SMS}
	\label{table:Goals}
	\renewcommand{\arraystretch}{1}  % Increase row height globally
	\begin{tabular}{p{1cm}p{6.8cm}}
		\toprule
		\textbf{Objective} & \textbf{Description}                                                                                                                                                                                                                                                                                                                      \\
		\midrule
		\textbf{G1}        & Classify works related to HTCondor universes according to their application and impact in the domains of distributed and parallel computing, HTC, software development, virtualization, microservices, computer networks, computational infrastructure, artificial intelligence, data analysis, and computational thinking, among others. \\
		\addlinespace[0.8em]
		\textbf{G2}        & Identify and categorize works related to HTCondor universes as a tool to strengthen essential university functions such as research, teaching, outreach, and industry.                                                                                                                                                                    \\
		\bottomrule
	\end{tabular}
\end{table}

% 4.1.2
%sub-subsection - research question
\subsubsection{Research Question}
For the construction of the research questions (RQs), the PICOC model \cite{Needleman20026, Petticrew2008systematic} was used. This model allows us to establish the aspects ``Population,'' ``Intervention,'' ``Comparison,'' ``Outcomes,'' and ``Context.'' This ensures situating the study in an appropriate environment and delivering value. See Table \ref{table:PICOC}.

Drawing on the PICOC model, two research questions (RQs) were formulated, as presented in Table~\ref{table:RQs}.

\begin{table}[htbp]
	\centering
	\caption{Objectives of the SMS}
	\label{table:PICOC}
	\renewcommand{\arraystretch}{1}  % Increase row height globally
	\begin{tabular}{p{1.4cm}p{6.4cm}}
		\toprule
		\textbf{Component}    & \textbf{Description}                                                                                                                                                                                                                                                                                                                                                                                                                                              \\
		\midrule
		\textbf{Population}   & Works related to HTCondor universes according to their application and impact in the domains of distributed and parallel computing, HTC, software development, virtualization and microservices, computer networks, computational infrastructure, artificial intelligence, data analysis, computational thinking, among others, which strengthen the core university functions of research, teaching, and outreach.                                               \\
		\addlinespace[0.8em]
		\textbf{Intervention} & Identification and classification of a set of works related to HTCondor universes according to their application and impact in the domains of distributed and parallel computing, HTC, software development, virtualization and microservices, computer networks, computational infrastructure, artificial intelligence, data analysis, computational thinking, among others, which strengthen the core university functions of research, teaching, and outreach. \\
		\addlinespace[0.8em]
		\textbf{Comparison}   & Documented project cases; fulfillment of inclusion and exclusion criteria; appearance in selected databases.                                                                                                                                                                                                                                                                                                                                                      \\
		\addlinespace[0.8em]
		\textbf{Outcomes}     & Taxonomy that organizes works related to HTCondor universes according to their application and impact in the domains of distributed and parallel computing, HTC, software development, virtualization and microservices, computer networks, computational infrastructure, artificial intelligence, data analysis, computational thinking, among others, which strengthen the core university functions of research, teaching, and outreach.                       \\
		\addlinespace[0.8em]
		\textbf{Context}      & HTCondor universes in domains of distributed and parallel computing, HTC, software development, virtualization and microservices, computer networks, computational infrastructure, artificial intelligence, data analysis, computational thinking, among others, which strengthen the core university functions of research, teaching, and outreach.                                                                                                              \\
		\bottomrule
	\end{tabular}
\end{table}

\begin{table*}[htbp]
	\centering
	\caption{Research questions of the SMS}
	\label{table:RQs}
	\renewcommand{\arraystretch}{1}  % Increase row height globally
	\begin{tabular}{p{1cm}p{1.7cm}p{6.8cm}p{6.8cm}}
		\toprule
		\textbf{Objective} & \textbf{Research Question} & \textbf{Description}                                                                                                                                                                                                                                                                                          & \textbf{Motivation}                                                                                                                                                                                                                                                                                                                                                             \\
		\midrule
		G1                 & RQ1                        & What works related to HTCondor universes have an impact in the domains of distributed and parallel computing, HTC, software development, virtualization and microservices, computer networks, computational infrastructure, artificial intelligence, data analysis, and computational thinking, among others? & To recognize how HTCondor universes impacting the domains of distributed and parallel computing, HTC, software development, virtualization and microservices, computer networks, computational infrastructure, artificial intelligence, data analysis, and computational thinking are structured, to identify their applications, and to determine their contextual motivation. \\
		\addlinespace[0.8em]
		G2                 & RQ2                        & What works related to HTCondor universes strengthen core university functions such as research, teaching, outreach, and industry?                                                                                                                                                                             & To recognize how HTCondor universes strengthening core university functions such as research, teaching, outreach, and industry are structured, to identify their applications, and to determine their contextual motivation.                                                                                                                                                    \\
		\bottomrule
	\end{tabular}
\end{table*}

% 4.1.3
%sub-subsection - metrics
\subsubsection{Metrics}
The SMS metrics were defined using a quantitative approach in accordance with the classification structure. The details of the metrics are shown in Table \ref{table:Metrics}. The criteria established restrict documents to a validity period of 44 years because of the scarcity of documentation on HTCondor and to provide readers with a broad range of articles, from foundational works to more recent studies. In addition, the study type was limiteed to primary studies indexed in recognized databases to ensure peer-review rigor.

\begin{table}[htbp]
	\centering
	\caption{Metrics of the SMS}
	\label{table:Metrics}
	\renewcommand{\arraystretch}{1}  % Increase row height globally
	\begin{tabular}{p{1cm}p{6.8cm}}
		\toprule
		\textbf{Metric} & \textbf{Description}                                                                            \\
		\midrule
		\textbf{M1}     & Number of works selected in the final phase of the SMS.                                         \\
		\addlinespace[0.8em]
		\textbf{M2}     & Popularity of each Universe in the works selected in the final phase.                           \\
		\addlinespace[0.8em]
		\textbf{M3}     & Percentage of works selected in the final phase with respect to the number of works considered. \\
		\addlinespace[0.8em]
		\textbf{M4}     & Percentage of works selected in the final phase, contributed by each database.                  \\
		\bottomrule
	\end{tabular}
\end{table}

% 4.1.4
%sub-subsection - research topics
\subsubsection{Topics}
The RQs and the PICOC model serve as the baseline for selecting the classification topics used in this study. The topics are as follows: Artificial Intelligence (AI), Cloud Computing, Containerization, Grid Computing, High Performance Computing (HPC), Java, Virtualization, Kubernetes, Networks, Parallelism, Docker, High Throughput Computing (HTC), Teaching, Research, and Industry.

% 4.1.5
%sub-subsection - inclusion and exclusion criteria
\subsubsection{Inclusion and Exclusion Criteria}
The inclusion and exclusion criteria defined for the study are shown in Table \ref{table:Criteria}.

\begin{table*}[htbp]
	\centering
	\caption{Inclusion and exclusion criteria of the SMS}
	\label{table:Criteria}
	\renewcommand{\arraystretch}{1}  % Increase row height globally
	\begin{tabular}{p{2.7cm}p{7cm}p{7cm}}
		\toprule
		\textbf{Category}   & \textbf{Inclusion Criteria}                                                                                                                        & \textbf{Exclusion Criteria}                                                                                                        \\
		\midrule
		Fields              & All.                                                                                                                                               & -                                                                                                                                  \\
		\addlinespace[0.8em]
		Type of publication & Research articles.                                                                                                                                 & Theses, book chapters, books, journals, \textit{proceedings}, \textit{papers}, and anything not covered by the inclusion criteria. \\
		\addlinespace[0.8em]
		Area or discipline  & Computer science, Engineering (in the ACM database it is assumed that all articles are related to these subjects since filtering is not possible). & Areas unrelated to computer science and engineering.                                                                               \\
		\addlinespace[0.8em]
		Period              & From 1980 to 2024.                                                                                                                                 & -                                                                                                                                  \\
		\addlinespace[0.8em]
		Language            & English.                                                                                                                                           & -                                                                                                                                  \\
		\bottomrule
	\end{tabular}
\end{table*}

A period of 44 years was defined in order to include the most relevant studies throughout the history of HTCondor, making it possible (as a consequence of the limited documentation available on this technology) for future researchers to conduct rigorous work over such a broad time frame and thereby ensure the value of the present SMS. In addition, the study type was limited to journal articles.

\begin{equation}
	\label{equation:SCI}
	SCI = \frac{C}{A}
\end{equation}

The third quality criterion used corresponds to the relationship of the studies with the research questions; this criterion is named IRRQ (\textit{Index of Relationship to Research Questions}) (see Formula \ref{equation:IRRQ}), where \textit{N} corresponds to the number of RQs to which the study is related. Subsequently, \textit{N} is divided by 2, as this is the number of research questions defined.

\begin{equation}
	\label{equation:IRRQ}
	IRRQ = \frac{N}{2}
\end{equation}

