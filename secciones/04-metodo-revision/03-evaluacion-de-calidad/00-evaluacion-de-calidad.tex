% subsection 4.3
\subsection{Stage 3: Quality Assessment}
According to~\cite{Ali-01}, the incorporation of a quality assessment is not mandatory in an SMS. However, such assessment could bring an SMS closer to a systematic review~\cite{Petersen-01}. Consequently, we sought to verify the relevance of the studies identified by the SMS objectives through this process. To carry out such assessment, the CVI (\textit{Content Value Index}), SCI (\textit{Study Citation Index}), and IRRQ (\textit{Index of Relationship to Research Questions}) indices defined during the planning stage were employed.

%sub-subsection 4.3.1
\subsubsection{Content Validity Assessment}
In this assessment, the content of the studies was analyzed to determine their value in the research context. For this purpose, the CVI index described in the planning stage was applied.

The rating of studies according to the CVI index was performed with the assistance of the SMS-Builder software. Subsequently, a frequency analysis was conducted to select the studies from the most significant quartile, that is, those with the highest CVI. This type of assessment is performed when identifying the \totalEtapaDos{} studies included in the SMS, whose results are presented in \hbox{Step 5: Study classification.}

%sub-subsection 4.3.2
\subsubsection{Index for Quality Assessment by Number of Citations}
The assessment of studies was performed by the team, taking into account the SCI index. We calculated this index with the support of the SMS-Builder software~\cite{sms-builder-repo} and citation data from Google Scholar. Subsequently, a frequency analysis was also conducted to select the studies from the most significant quartile, that is, those with the highest SCI.

%sub-subsection 4.3.3
\subsubsection{Index for the Evaluation of the Relationship of Studies to Research Questions}
This quality assessment uses the IRRQ index, described in the planning stage (Section~\ref{sec:planeacion}). In this regard, the selected studies were analyzed to establish their relationship with the classification topics.

Through this process, the SMS-Builder software~\cite{sms-builder-repo} was configured to establish the direct relationship of the studies with the research questions proposed in this SMS. As in the previous cases, a frequency analysis was also performed, where the quartile of most significant studies was selected, that is, those with the highest IRRQ.