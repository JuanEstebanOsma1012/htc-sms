% subseccion 4.3
%
\subsection{Etapa 3: Evaluación de Calidad}
Según \cite{Ali-01} la incorporación de una evaluación de la calidad no es obligatoria en una SMS. No obstante, dicha evaluación podría aproximar una SMS a una revisión sistemática \cite{Petersen-01}. Por consiguiente, se buscó verificar la pertinencia de los estudios identificados por los objetivos de la SMS mediante este proceso. Para llevar a cabo dicha evaluación, se emplearon los índices CVI (\textit{Content Value Index}), SCI (\textit{Study Citation Index}) e IRRQ (\textit{Index of Relationship to Research Questions}) definidos durante la fase de planificación.

%sub-subseccion 4.3.1
\subsubsection{Evaluación de la Validez del Contenido}
En esta evaluación, se analizó el contenido de los estudios para determinar su valor en el contexto de la investigación. Para ello, se aplicó el índice CVI descrito en la fase de planificación.

La calificación de los estudios de acuerdo con el índice CVI se realizó con la ayuda del Software SMS-Builder. Posteriormente, se llevó a cabo un análisis de frecuencia para seleccionar los estudios del cuartil más significativo, es decir, aquellos con un CVI más alto. Este tipo de evaluación se realiza al identificar los \totalEtapaDos{} estudios incluidos en la SMS, cuyos resultados se presentan en el \hbox{Paso 5: Clasificación de estudios.}


%sub-subseccion 4.3.2
\subsubsection{Índice para la Evaluación de Calidad por Número de Citas}
La evaluación de los estudios fue realizada por el equipo, teniendo en cuenta el índice SCI. Calculamos este índice con el apoyo del software SMS-Builder \cite{sms-builder-repo} y los datos de citas de Google Scholar. Posteriormente, también se llevó a cabo un análisis de frecuencia para seleccionar los estudios del cuartil más significativo, es decir, aquellos con un SCI más alto.


%sub-subseccion 4.3.3
\subsubsection{Índice para la Evaluación de la Relación de los Estudios con las Preguntas de Investigación}
Esta evaluación de calidad utiliza el índice IRRQ, descrito en la fase de planificación (Sección \ref{sec:planeacion}). En este sentido, se analizaron los estudios seleccionados para establecer su relación con los temas de clasificación.

Mediante este proceso, se configuró el software SMS-Builder \cite{sms-builder-repo} para establecer la relación directa de los estudios con las preguntas de investigación propuestas en esta SMS. Al igual que en los casos anteriores, también se realizó un análisis de frecuencia, donde se seleccionó el cuartil de estudios más significativo, es decir, aquellos con un IRRQ más alto.
