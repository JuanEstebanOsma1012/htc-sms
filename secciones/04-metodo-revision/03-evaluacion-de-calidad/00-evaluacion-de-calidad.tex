% subseccion 4.3
\subsection{Etapa 3: Evaluación de Calidad}
\mbox{}\\
De acuerdo con~\cite{8747000}, no es necesario incorporar una evaluación de calidad en el mapeo sistemático. Sin embargo, dicha evaluación aproxima el SMS a una revisión sistemática~\cite{10.1145/2601248.2601268}. Por lo tanto, se busca verificar la relevancia de los estudios identificados en función de los objetivos del SMS mediante un proceso de evaluación de calidad. Se usó el CVI \textit{(Content Value Index)}, SCI \textit{(Study citation index)} y IRRQ \textit{(Index of Relationship to Research Questions)} definidos en la fase de planeación para evaluar la calidad de los estudios.\\

%sub-subseccion 4.3.1
\subsubsection{Evaluación de la Validez del Contenido}
\mbox{}\\
Esta evaluación consiste en analizar el contenido de los artículos para determinar su valor en el contexto de la investigación. Para ello, se utilizó el CVI, definido en la fase de planeación.
Los estudios se ordenaron de acuerdo con este índice, considerando un número impar de evaluadores. De esta manera, se evitaron empates, ya que un número impar siempre permite tomar una decisión mayoritaria. Además, este procedimiento condujo a un análisis de frecuencia para seleccionar el cuartil más significativo, es decir, aquel con el CVI más alto. \\
La evaluación se desarrolló en dos rondas. La primera se ejecutó durante el proceso de selección de la línea base en la estrategia de búsqueda de bola de nieve. Una vez identificados los \textbf{226} estudios incluidos en el SMS, se evaluó por segunda vez, cuyos resultados se presentan en el paso 5:~\hyperref[sec:clasificacion-estudios]{Clasificación de estudios}.\\

%sub-subseccion 4.3.2
\subsubsection{Índice para la Evaluación de Calidad por Número de Citas}
\mbox{}\\
Esta evaluación de los estudios se realizó considerando el índice SCI. Su calculó se efectuó con el apoyo del software SMS-Builder~\cite{candela2020smsbuilder} y los datos de citación proporcionados por Google Scholar. A continuación, se llevó a cabo un análisis de frecuencia para obtener el cuartil más significativo, correspondiente a los estudios con el SCI más alto. \\

%sub-subseccion 4.3.3
\subsubsection{Índice para la Evaluación de la Relación de los Estudios con las Preguntas de Investigación}
\mbox{}\\
Esta evaluación de la calidad usa el índice IRRQ, descrito en la fase de planeación (Sección~\ref{subsec:planeacion}). En este sentido, se analizaron los estudios con el propósito de establecer su relación con los tópicos de clasificación, los cuales poseen una correspondencia directa con las preguntas de investigación.\\
Para este proceso, se empleó el software SMS-Builder~\cite{candela2020smsbuilder}, con el fin de determinar la relación entre los estudios y las preguntas de investigación planteadas en este SMS. De manera similar, se efectuó un análisis de frecuencia para seleccionar el cuartil más significativo, correspondiente a los estudios con el IRRQ más alto.