% subseccion 4.3
\subsection{Etapa 3: Evaluación de Calidad}
\mbox{}\\
De acuerdo a~\cite{8747000}, no es necesario incorporar una evaluación de calidad en el mapeo sitemático. Sin embargo, una evaluación de calidad podría acercar el SMS a una revisión sistemática~\cite{10.1145/2601248.2601268}. Por lo tanto, se busca verificar la relevancia de los estudios identificados por los objetivos del SMS a través de la evaluación de calidad. Se usó el CVI \textit{(Content Value Index)}, SCI \textit{(Study citation index)} y IRRQ \textit{(Index of Relationship to Research Questions)} definidos en la fase de planeación con el objetivo de la evaluación de calidad.\\

%sub-subseccion 4.3.1
\subsubsection{Evaluación de la Validez del Contenido}
\mbox{}\\
Esta evaluación involucra analizar el contenido de los artículos para determinar su valor en el contexto de la investigación. Para esto se usó el CVI descrito en la fase de planeación. \\
Los estudios se ordenaron respecto al índice CVI, con un número impar de evaluadores. De esta manera, se evitó los empates en este índice, ya que, si el número de evaluadores es impar siempre se puede tomar una decisión mayoritaria. Además, esto conduce a un análisis de frecuencia para seleccionar el cuartil más significativo, o lo que es lo mismo, con el CVI más alto. \\
Esta evaluación se realizó en dos rondas. La primera vez fue en en el proceso de selección de la línea base en la estrategia de búsqueda de bola de nieve. La segunda vez fue luego de identificar los 226 estudios incluidos en el SMS, cuyos resultados se muestran en el paso 5:~\hyperref[sec:clasificacion-estudios]{Clasificación de estudios}.\\

%sub-subseccion 4.3.2
\subsubsection{Índice para la Evaluación de Calidad por Número de Citas}
\mbox{}\\
Esta evaluación de los estudios fue hecha teniendo en cuenta el índice SCI. Se calculó este índice apoyado por el software SMS-Builder~\cite{candela2020smsbuilder} y los datos de citación proporcionados por Google Scholar. Paso seguido, también se realizó un análisis de frecuencia para obtener el cuartil más significativo de los estudios: Aquellos con el SCI más alto \\

%sub-subseccion 4.3.3
\subsubsection{Índice para la Evaluación de la Relación de los Estudios con las Preguntas de Investigación}
\mbox{}\\
Esta evaluación de la calidad usa el índice IRRQ descrito en la fase de planeación (Sección~\ref{subsec:planeacion}). En este sentido, se analizó los estudios para establecer su relación con los tópicos de clasificación. Además de que los tópicos también tienen una relación directa con las preguntas de investigación.\\
A través de este proceso, se usó el software SMS-Builder~\cite{candela2020smsbuilder} para establecer la relación entre los estudios y las preguntas de investigación propuestas en este SMS. Como en el caso anterior, se realizó un análisis de frecuencia donde se seleccionó el cuartil con los estudios más significativo: Aquellos con el IRRQ más alto.