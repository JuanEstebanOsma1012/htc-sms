% subseccion 4.6
\subsection{Etapa 6: Resultados}
Este apartado presenta los resultados obtenidos a través de la ejecución del protocolo de búsqueda de la SMS. Inicialmente, se ofrece una descripción general de los datos correspondientes a los estudios primarios seleccionados SPS.

Esta descripción abarca aspectos como el origen de los estudios, el año de publicación, la estrategia de búsqueda utilizada para su localización, los índices de calidad, la relación con las preguntas de investigación (RQs) y sus respectivos temas, y las palabras clave. Por último, se presenta una descripción de una nube de palabras generada a partir de las palabras clave de los SPS.

Es importante recordar que la asociación entre los SPS y los temas se estableció mediante la clasificación detallada en la Sección \ref{subsec:clasificacion-de-estudios}. De manera inherente, las RQs también se asociaron, dado que los temas tienen una estructura específica dentro de ellas.

Asimismo, se identificó la asociación entre las principales palabras clave de la SMS con los títulos de las secciones, los resúmenes y las palabras clave en todos los SPS.

%sub-subseccion 4.6.1
\subsubsection{Descripción General de los SPSs (Estudios Primarios Seleccionados)}
El proceso de busqueda y filtrado de estudios resultó en la identificación de 114 SPSs como se muestra en la Tabla \ref{table:selected_primary_studies}.

Estos SPSs vienen de bases de datos digitales, que identificamos en el SMS a través de diferentes estrategias. La Figura \ref{} muestra el numero de estudios identificados a través de varias fuentes y los detalles de las estrategias de busqueda en bases de datos y bola de nieve. Independientemente de la fuente, XX.XX\% de los SPSs fueron encontrados a través de la estategia de búsqueda en bases de datos, mientras que el XX.XX\% restante se identificó mediante la estrategia de bola de nieve.

Sin importar la estrategia de búsqueda en bases de datos, la Figura \ref{} muestra que de los XXX estudios encontrados a través de la estrategia de búsqueda en bases de datos, el XX.XX\% fueron encontrados en la base de datos X, el XX.XX\% en la base de datos Y, y el XX.XX\% en la base de datos Z. Mirando los XX SPSs identificados por la estrategia de bola de nieve, el XX.XX\% fueron encontrados mirando hacia atrás y el XX.XX\% mirando hacia adelante.

Considerando las relaciones de los SPSs con las RQs, la Figura \ref{} muestra que el XX.XX\% de los SPSs están relacionados con la RQ1, mientras que el XX.XX\% con la RQ2.

También, la Figura \ref{} muestra que el XX.XX\% de los estudios están relacionados exclusivamente con la RQ1, mientras que el XX.XX\% exclusivamente con la RQ2. Finalmente, XX.XX\% de los SPSs están relacionados con ambas RQs simultaneamente.

La Figura \ref{} muestra los topicos definidos en la etapa de planeación por cada pregunta de investigación y la cantidad de SPSs relacionados con cada topico de forma inclusiva. Es esencial tener en cuenta que un SPS puede estar relacionado con varios topicos simultaneamente. En este sentido, la Figura \ref{} los XX topicos relacionados con la RQ1, de los que el topicos "XXXXX" es el más frecuente con XX.XX\%. En contraste, el topico con la menor frecuencia es XXXXX con XX.XX\%. Por otro lado, la Figura \ref{} muestra los XX topicos relacionados con la RQ2, donde el topico "XXXXX" es el más frecuente con XX.XX\%, mientras que el topico con la menor frecuencia es XXXXX con XX.XX\%.

La Figura \ref{} tambien muestra los tres topicos relacionados con la pregunta de investigación RQ2, en la que el topico "XXXXX" es el más frecuente con XX.XX\%, mientras que el topico con la menor frecuencia es XXXXX con XX.XX\%.

Se recalca que el periodo de busqueda va desde XXXX hasta XXXX; en este sentido, la Figura \ref{} muestra un decremento en la cantidad de estudios publicados entre XXXX y XXXX. Ver Figura \ref{}. En contraste en XXXX hubo un incremento respecto a XXXX con un total de XX SPSs. El menor número de SPSs fue en el año XXXX.

De acuerdo al indice SCI, este SMS presenta en XXXX y XXXX, el numero más grande de SPSs en el cuartil más representativo. Con respecto al indice CVI, el SMS muestra regularidad en la cantidad de SMSs por año, excepto por XXXX y XXXX. Considerando el indice IRRQ, el SMS indica que ningún SPS fué publicado para XXXX. De otro modo, los años XXXX y XXXX registraron la mayor cantidad de estudios relacionados directamente con las dos RQs.

La Figura \ref{} muestra la cantidad de SPSs clasificados por pregunta de investigación e incluyendo sus respectivos topicos, adicionalmente se consideran los indices de calidad.

En este sentido, este estudio indica que para la RQ1, el topico "XXXXX" registra la menor cantidad con XX SPSs, equivalente al XX.XX\% de los 114 SPSs. En contraste, el topico "XXXXX" es el más frecuente con XX SPSs, equivalente al XX.XX\% de los 114 SPSs. Por otro lado, para la RQ2, los topicos más frecuentes son "XXXXX" y "XXXXX", con XX y XX SPSs respectivamente, equivalentes al XX.XX\% y XX.XX\% de los 114 SPSs. En contraste, el topico "XXXXX" es el menos frecuente con XX SPSs, equivalente al XX.XX\% de los 114 SPSs.

Respecto a los indices CVI, SCI y IRRQ, identificamos los topicos "XXXXX", "XXXXX" y "XXXXX" como las categorias que más estudios contienen.

La Figura \ref{} muestra un analisis cruzado entre los SPSs, palabras clave y sinonimos. La palabra clave "XXXXX" permitió la identificación de XX SPSs. Respecto a la palabra clave "XXXXX" permitió la identificación de XX SPSs. Por otro lado la palabra clave "XXXXX" tiene la menor contribución identificando SPSs.

%sub-subseccion 4.6.2
\subsubsection{Visualización de Nube de Palabras}
Otro resultado obtenido de los 114 SPSs del SMS, es la construcción de una nube de palabras. Con este tipo de grafico, se busca identificar las palabras y conceptos más frecuentes y que tienen la mayor importancia. La Figura \ref{} muestra la nube de palabras generada con la herramienta XXXXX, cuya construcción se realizó a partir de las palabras clave de los SPSs.

Dentro de las palabras claves más frecuentes, se destacan "XXXXX", "XXXXX" y "XXXXX", con una contribución del XX.XX\% en conjunto. El segundo nivel de relevancia en terminos de nube de palabras corresponden a "XXXXX", "XXXXX" y "XXXXX", con una contribución del XX.XX\%. Finalmente, las palabras menos frecuentes incluyen "XXXXX", "XXXXX" y "XXXXX", con una contribución del XX.XX\%.
