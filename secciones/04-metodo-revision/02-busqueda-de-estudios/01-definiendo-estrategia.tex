%sub-subseccion 4.2.1 
\subsubsection{Definiendo la Estrategia de Búsqueda}\label{subsubsec:estrategia-busqueda}

Para desarrollar este SMS, se ha implementado una metodología híbrida. Esta metodología tiene como objetivo conseguir una cantidad más amplia de estudios indexados y procedentes de múltiples fuentes, superando los resultados que ofrecen en las bases de datos. De esta manera, integramos dos técnicas de búsqueda. La técnica inicial se denomina ``Búsqueda en bases de datos'' la cual consiste en realizar una búsqueda automatizada dentro de bases de datos académicas indexadas~\cite{Jalai-01}. La técnica secundaria recibe el nombre de \textit{Snowballing} o bola de nieve y representa un procedimiento manual fundamentado en una previa recolección de textos iniciales para identificar investigaciones adicionales mediante sus bibliografías y citaciones~\cite{Jalai-01,Goodman-01}.

Para apoyar el proceso llevado a cabo en este SMS, hemos utilizado los siguientes elementos: a) Bases de datos académicas. b) El software SMS-Builder desarrollado por Candela et al.~\cite{Candela2022100935} diseñado específicamente para facilitar la construcción de estudios de mapeo sistemático. c) Herramientas para apoyar la gestión de referencias como Mendeley y la busqueda de las mismas como Google Scholar.